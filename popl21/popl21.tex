\nonstopmode
%% For double-blind review submission, w/o CCS and ACM Reference (max submission space)
\documentclass[acmsmall,review,anonymous]{acmart}\settopmatter{printfolios=true,printccs=false,printacmref=false}
%% For double-blind review submission, w/ CCS and ACM Reference
%\documentclass[acmsmall,review,anonymous]{acmart}\settopmatter{printfolios=true}
%% For single-blind review submission, w/o CCS and ACM Reference (max submission space)
%\documentclass[acmsmall,review]{acmart}\settopmatter{printfolios=true,printccs=false,printacmref=false}
%% For single-blind review submission, w/ CCS and ACM Reference
%\documentclass[acmsmall,review]{acmart}\settopmatter{printfolios=true}
%% For final camera-ready submission, w/ required CCS and ACM Reference
%\documentclass[acmsmall]{acmart}\settopmatter{}


%% Journal information
%% Supplied to authors by publisher for camera-ready submission;
%% use defaults for review submission.
\acmJournal{PACMPL}
\acmVolume{5}
\acmNumber{POPL} % CONF = POPL or ICFP or OOPSLA
\acmArticle{1}
\acmYear{2021}
\acmMonth{1}
\acmDOI{} % \acmDOI{10.1145/nnnnnnn.nnnnnnn}
\startPage{1}

%% Copyright information
%% Supplied to authors (based on authors' rights management selection;
%% see authors.acm.org) by publisher for camera-ready submission;
%% use 'none' for review submission.
\setcopyright{none}
%\setcopyright{acmcopyright}
%\setcopyright{acmlicensed}
%\setcopyright{rightsretained}
%\copyrightyear{2018}           %% If different from \acmYear

%% Bibliography style
\bibliographystyle{ACM-Reference-Format}
%% Citation style
%% Note: author/year citations are required for papers published as an
%% issue of PACMPL.
\citestyle{acmauthoryear}   %% For author/year citations


%%%%%%%%%%%%%%%%%%%%%%%%%%%%%%%%%%%%%%%%%%%%%%%%%%%%%%%%%%%%%%%%%%%%%%
%% Note: Authors migrating a paper from PACMPL format to traditional
%% SIGPLAN proceedings format must update the '\documentclass' and
%% topmatter commands above; see 'acmart-sigplanproc-template.tex'.
%%%%%%%%%%%%%%%%%%%%%%%%%%%%%%%%%%%%%%%%%%%%%%%%%%%%%%%%%%%%%%%%%%%%%%


%% Some recommended packages.
\usepackage{booktabs}   %% For formal tables:
                        %% http://ctan.org/pkg/booktabs
\usepackage{subcaption} %% For complex figures with subfigures/subcaptions
                        %% http://ctan.org/pkg/subcaption

\usepackage{stmaryrd}
\usepackage{xspace}
\usepackage{xypic}

\usepackage{dashrule}
\newcommand{\dashruler}{\hdashrule[0.5ex]{\textwidth}{0.2pt}{1ex}}
\newcommand{\ruler}{\rule{\textwidth}{0.2pt}}

% % https://tex.stackexchange.com/questions/46828/how-to-highlight-important-parts-with-a-gray-background
% \usepackage[breakable, theorems, skins]{tcolorbox}
% \tcbset{enhanced}

% \DeclareRobustCommand{\graytextbox}[2][gray!20]{%
% \begin{tcolorbox}[   %% Adjust the following parameters at will.
%         breakable,
%         left=0pt,
%         right=0pt,
%         top=0pt,
%         bottom=0pt,
%         colback=#1,
%         colframe=#1,
%         width=\dimexpr\textwidth\relax,
%         enlarge left by=0mm,
%         boxsep=5pt,
%         arc=0pt,outer arc=0pt,
%         ]
%         #2
% \end{tcolorbox}
% }

% https://tex.stackexchange.com/questions/502652/define-tcolorbox-in-math-mode
\usepackage{tcolorbox}
\newtcbox{\grayboxtext}{on line,colback=gray!50,colframe=gray!50,size=fbox,arc=0pt,boxrule=0pt}%,beforeafter skip=0pt,leftright skip=0pt}
\newcommand{\graybox}[1]{\grayboxtext{$#1$}}

\theoremstyle{remark}
\newtheorem{remark}{Remark}

\begin{document}

%% Title information
\title[Graded CBPV]{Graded Call-By-Push-Value}         %% [Short Title] is optional;
                                        %% when present, will be used in
                                        %% header instead of Full Title.
%\titlenote{with title note}             %% \titlenote is optional;
                                        %% can be repeated if necessary;
                                        %% contents suppressed with 'anonymous'
%\subtitle{Subtitle}                     %% \subtitle is optional
%\subtitlenote{with subtitle note}       %% \subtitlenote is optional;
                                        %% can be repeated if necessary;
                                        %% contents suppressed with 'anonymous'


%% Author information
%% Contents and number of authors suppressed with 'anonymous'.
%% Each author should be introduced by \author, followed by
%% \authornote (optional), \orcid (optional), \affiliation, and
%% \email.
%% An author may have multiple affiliations and/or emails; repeat the
%% appropriate command.
%% Many elements are not rendered, but should be provided for metadata
%% extraction tools.

%% Author with single affiliation.
\author{Andreas Abel}
%\authornote{with author1 note}          %% \authornote is optional;
                                        %% can be repeated if necessary
\orcid{0000-0003-0420-4492}             %% \orcid is optional
\affiliation{
  \department{Department of Computer Science and Engineering}
  \institution{Gothenburg University}
  \streetaddress{Rännvägen 6b}
  \city{Göteborg}
%  \state{State1}
  \postcode{41296}
  \country{Sweden}
}
\email{andreas.abel@@gu.se}          %% \email is recommended

% %% Author with two affiliations and emails.
% \author{First2 Last2}
% \authornote{with author2 note}          %% \authornote is optional;
%                                         %% can be repeated if necessary
% \orcid{nnnn-nnnn-nnnn-nnnn}             %% \orcid is optional
% \affiliation{
%   \position{Position2a}
%   \department{Department2a}             %% \department is recommended
%   \institution{Institution2a}           %% \institution is required
%   \streetaddress{Street2a Address2a}
%   \city{City2a}
%   \state{State2a}
%   \postcode{Post-Code2a}
%   \country{Country2a}                   %% \country is recommended
% }
% \email{first2.last2@inst2a.com}         %% \email is recommended
% \affiliation{
%   \position{Position2b}
%   \department{Department2b}             %% \department is recommended
%   \institution{Institution2b}           %% \institution is required
%   \streetaddress{Street3b Address2b}
%   \city{City2b}
%   \state{State2b}
%   \postcode{Post-Code2b}
%   \country{Country2b}                   %% \country is recommended
% }
% \email{first2.last2@inst2b.org}         %% \email is recommended


%% Abstract
%% Note: \begin{abstract}...\end{abstract} environment must come
%% before \maketitle command
\begin{abstract}
Call-by-push-value (CBPV) is a simply typed lambda calculus that polarizes
types into value and computation types and can thus express both
call-by-name and call-by-value evaluation in the presence of effects.
Semantically, effects are modeled by a monad, and computation types as
algebras over this monad.
%
Effect type systems usually express more information than the presence
of an effect; often effects are categorized by preordered monoid
where the monoid operation represents accumulation of effects and the
order expresses effect subsumption, in analogy to subtyping.
In this work, a \emph{graded} version of CBPV is presented where the typing
of computations likens effect typing.  Semantically, computation types
are then represented as graded monad algebras.
%
Observing that the value types of CBPV can be interpreted as comonad coalgebras, we further present a version of CBPV that has coeffects graded by a preordered semiring.  Value types and contexts are interpreted as graded comonad coalgebras, allowing resource-aware interpretations of CBPV.
%
Finally, we combine the two systems into a fully graded version of CBPV where both effects and coeffects are graded.  Thus turns out to be possible without specifying an interaction between effects and coeffects.
\end{abstract}


%% 2012 ACM Computing Classification System (CSS) concepts
%% Generate at 'http://dl.acm.org/ccs/ccs.cfm'.
 \begin{CCSXML}
<ccs2012>
<concept>
<concept_id>10003752.10003790.10011740</concept_id>
<concept_desc>Theory of computation~Type theory</concept_desc>
<concept_significance>500</concept_significance>
</concept>
<concept>
<concept_id>10003752.10010124.10010125.10010130</concept_id>
<concept_desc>Theory of computation~Type structures</concept_desc>
<concept_significance>500</concept_significance>
</concept>
<concept>
<concept_id>10003752.10010124.10010138.10010142</concept_id>
<concept_desc>Theory of computation~Program verification</concept_desc>
<concept_significance>300</concept_significance>
</concept>
<concept>
<concept_id>10003752.10010124.10010131.10010134</concept_id>
<concept_desc>Theory of computation~Operational semantics</concept_desc>
<concept_significance>100</concept_significance>
</concept>
</Cs2012>
\end{CCSXML}

\ccsdesc[500]{Theory of computation~Type theory}
\ccsdesc[500]{Theory of computation~Type structures}
\ccsdesc[300]{Theory of computation~Program verification}
\ccsdesc[100]{Theory of computation~Operational semantics}
%% End of generated code

%% Keywords
%% comma separated list
\keywords{effects, coeffects, call-by-name, call-by-value, linear types}  %% \keywords are mandatory in final camera-ready submission


%% \maketitle
%% Note: \maketitle command must come after title commands, author
%% commands, abstract environment, Computing Classification System
%% environment and commands, and keywords command.
\maketitle


%\newcommand{}{\ensuremath{}}
\newcommand{\bla}{\ensuremath{\mbox{$$}}}
\newcommand{\ie}{\emph{i.e.}}
\newcommand{\eg}{\emph{e.g.}}
\newcommand{\Eg}{\emph{E.g.}}
\newcommand{\loccit}{\emph{loc.\,cit.}\xspace}
\newcommand{\cf}{cf.\ }
\newcommand{\den}[2][]{\llbracket#2\rrbracket^{#1}}
\newcommand{\dent}[1]{\llparenthesis#1\rrparenthesis}
\newcommand{\To}{\ensuremath{\Rightarrow}}
%\newcommand{\todot}{\stackrel\cdot\to}
\newcommand{\todot}{\mathbin{\dot{\to}}}
\newcommand{\bN}{\ensuremath{\mathbb{N}}}
\newcommand{\dom}{\mathop{\mathrm{dom}}\nolimits}
\newcommand{\Pot}[1]{\mathcal{P}\,#1}
%\newcommand{\defiff}{:\iff}
\newcommand{\defiff}{\mathrel{{{:}{\Longleftrightarrow}}}}
\newcommand{\subst}[3]{#3[#1/#2]}
\newcommand{\qsubst}[4]{#4[#1/#2#3]}

\newcommand{\denp}{\den[+]}
\newcommand{\denn}{\den[-]}
\newcommand{\Denp}[2]{\denp{#1}_{#2}}
\newcommand{\Denn}[2]{\denn{#1}_{#2}}
\newcommand{\Den}[2]{\den{#1}_{#2}}
\newcommand{\Denpar}[2]{\Den{#1}{(#2)}}
\newcommand{\denv}[2]{\dent{#1}_{#2}}
\newcommand{\denvp}[2]{\dent{#1}_{(#2)}}

\newcommand{\tEff}{\ensuremath{\mathsf{Eff}}}
\newcommand{\Eff}{\ensuremath{\mathsf{Eff}}}
\newcommand{\bu}{\ensuremath{\bullet}}
\newcommand{\Ge}{\ensuremath{\varepsilon}}
\newcommand{\Ga}{\ensuremath{\alpha}}
\newcommand{\Gd}{\ensuremath{\delta}}
\newcommand{\Gg}{\ensuremath{\gamma}}
\newcommand{\Gl}{\ensuremath{\lambda}}
\newcommand{\Gr}{\ensuremath{\rho}}
\newcommand{\GG}{\ensuremath{\Gamma}}
\newcommand{\GS}{\ensuremath{\Sigma}}
\newcommand{\R}{\mathsf{R}}
\newcommand{\I}{\mathsf{I}}
\newcommand{\LL}{\mathcal{L}}
\newcommand{\W}{\mathsf{W}}
\newcommand{\M}{\mathsf{M}}
\newcommand{\wleq}{\sqsubseteq}
\newcommand{\winf}{\sqcap}
\newcommand{\wsup}{\sqcup}
%\newcommand{\winf}{\bigsqcap}
\newcommand{\emp}{0_{\W}}
% \newcommand{\emp}{\varepsilon}
% \newcommand{\emp}{\emptyset}

\newcommand{\CC}{\ensuremath{\mathcal{C}}}
\newcommand{\EE}{\ensuremath{\mathcal{E}}}
\newcommand{\Hom}[3][\CC]{#1\,(#2,\,#3)}
\newcommand{\SET}{\mathsf{SET}}
\newcommand{\CPO}{\mathsf{CPO}}
\newcommand{\T}[1][]{\mathsf{T}_{#1}}
\newcommand{\D}[1][]{\mathsf{D}_{#1}}
\newcommand{\trun}{\mathsf{run}}
\newcommand{\run}[1][]{\mathsf{run}_{#1}}
\newcommand{\texpose}{\mathsf{expose}}
\newcommand{\expose}[1][]{\texpose_{#1}}
\newcommand{\tA}{\mathsf{A}}
\newcommand{\tB}{\mathsf{B}}
\newcommand{\A}{\mathsf{A}}
\newcommand{\B}{\mathsf{B}}
\newcommand{\tExc}{\mathsf{Exc}}
\newcommand{\Exc}{\mathsf{Exc}}
\newcommand{\String}{\mathsf{String}}
\newcommand{\StringLe}[1]{\mathsf{String}{\leq}\,#1}
\newcommand{\tfmap}{\mathsf{fmap}}
\newcommand{\fmap}[2][]{\tfmap_{#1}\,#2}
\newcommand{\fmapt}[2][]{\tfmap_{#1}#2}
\newcommand{\treturn}{\mathsf{return}}
\newcommand{\return}[1][]{\treturn_{#1}}
\newcommand{\tjoin}{\mathsf{join}}
\newcommand{\join}[1][]{\mathsf{join}_{#1}}
\newcommand{\append}{\ensuremath{\mathbin{\mathtt{++}}}}
\newcommand{\tcast}{\mathsf{cast}}
\newcommand{\tdist}{\mathsf{dist}}
\newcommand{\Id}{\mathsf{Id}}
\newcommand{\tid}[1][]{\mathsf{id}_{#1}}
\newcommand{\teval}{\mathsf{eval}}
\newcommand{\tstrength}{\mathsf{strength}}
\newcommand{\strengthl}{\tstrength^{\tl}}
\newcommand{\strengthr}{\tstrength^{\tr}}
\newcommand{\tl}{\mathsf{l}}
\newcommand{\tr}{\mathsf{r}}
\newcommand{\tcurry}{\mathsf{curry}}
\newcommand{\comp}{\circ}
\newcommand{\tweaken}{\mathsf{weaken}}
\newcommand{\tdrop}[1][]{\mathsf{drop}_{#1}}
%\newcommand{\drop}[1][]{\tdrop_{#1}} % Name clash
\newcommand{\tcontract}{\mathsf{contract}}
\newcommand{\contract}[1][]{\tcontract_{#1}}

% Types
\newcommand{\tTy}{\mathsf{Ty}}
\newcommand{\Ty}[1]{\tTy^{#1}}
\newcommand{\PTy}{\Ty+}
\newcommand{\NTy}{\Ty-}
\newcommand{\Cxt}{\mathsf{Cxt}}
\newcommand{\RCxt}{\mathsf{RCxt}}
\newcommand{\cempty}{\emptyset}
\newcommand{\sempty}{\emptyset}
\newcommand{\sext}[3]{#1,#3/#2}
\newcommand{\extr}[2]{#1.#2}
\newcommand{\ext}[3]{#1.#2{:}#3}
\newcommand{\qext}[5]{#1#2.#3#4{:}#5}
\newcommand{\gqext}[5]{\graybox{#1}#2.\graybox{#3}#4{:}#5}


\newcommand{\sumt}   [2]{\ensuremath{\Sigma_{#1} #2}}
\newcommand{\tupt}   [2]{\ensuremath{\mathop\otimes_{#1} #2}}
\newcommand{\rect}   [2]{\ensuremath{\Pi_{#1} #2}}
\newcommand{\sumty}  [3]{\ensuremath{\Sigma_{#1:#2} #3_{#1}}}
\newcommand{\tupty}  [3]{\ensuremath{\mathop\otimes_{#1:#2} #3_{#1}}}
\newcommand{\recty}  [3]{\ensuremath{\Pi_{#1:#2} #3_{#1}}}
\newcommand{\thunkty}[2]{\ensuremath{[#1]#2}}
\newcommand{\compty} [2]{\ensuremath{\langle#1\rangle#2}}

% Terms
\newcommand{\tTm}{\mathsf{Tm}}
\newcommand{\Tm}[1]{\tTm^{#1}}
\newcommand{\PTm}{\Tm+}
\newcommand{\NTm}{\Tm-}
\newcommand{\tthunk}{\mathsf{thunk}}
\newcommand{\thunk}[1]{\tthunk\,#1}
\newcommand{\tforced}{\mathsf{forced}}
\newcommand{\tforce}{\mathsf{force}}
\newcommand{\force}[1]{\tforce\,#1}
\newcommand{\tin}[1]{\mathsf{in}_{#1}}
\newcommand{\inj}[2]{\tin{#1}\,#2}
\newcommand{\ttup}[1]{\mathsf{tup}^{#1}}
\newcommand{\tup}[2]{\ttup{#1}\,{#2}}
\newcommand{\ptup}[1]{\ttup\,{#1}}
% \newcommand{\ptup}[1]{\ttup{+}\,{#1}}
\newcommand{\ntup}[1]{\ttup{-}\,{#1}}
\newcommand{\tret}{\mathsf{ret}}
%\newcommand{\vlet}[3]{#1\,{=}\,#2;\,#3}
\newcommand{\ret}[1]{\tret\,#1}
\newcommand{\vlet}[3]{#2\,{\mathsf{be}}\,#1.\,#3}
% \newcommand{\let}[3]{#1\,{\leftarrow}\,#2;\,#3}
\newcommand{\bind}[3]{#2\,{\mathsf{to}}\,#1.\,#3}
% \newcommand{\bind}[3]{\ret #1\,{\leftarrow}\,#2;\,#3}
\newcommand{\tsplit}{\mathsf{split}}
\newcommand{\splits}[3]{#1\,{\tsplit}\,#2.\,#3}
\newcommand{\tcases}{\mathsf{cases}}
\newcommand{\case}[2]{#1\,{\tcases}\,\{#2\}}
\newcommand{\caser}[2]{#1\,{\tcases}\,#2}
\newcommand{\qlet}[4]{#3\,{\mathsf{be}}\,#1\,#2.\,#4}
\newcommand{\qbind}[4]{#3\,{\mathsf{to}}\,#1\,#2.\,#4}
\newcommand{\qsplits}[4]{#2\,{\tsplit}\,#1\,#3.\,#4}
\newcommand{\qforced}[4]{#2\,{\tforced}\,#1\,#3.\,#4}
\newcommand{\qcase}[3]{#2\,{\tcases}\,\{#1\,#3\}}
\newcommand{\gqlet}[4]{#3\,{\mathsf{be}}\,\graybox{#1}#2.\,#4}
\newcommand{\gqbind}[4]{#3\,{\mathsf{to}}\,\graybox{#1}#2.\,#4}
\newcommand{\gqsplits}[4]{#2\,{\tsplit}\,\graybox{#1}#3.\,#4}
\newcommand{\gqforced}[4]{#2\,{\tforced}\,\graybox{#1}#3.\,#4}
\newcommand{\gqcase}[3]{#2\,{\tcases}\,\{\graybox{#1}#3\}}
% \newcommand{\qlet}[4]{#1\,#3\,{\mathsf{be}}\,#2.\,#4}
% \newcommand{\qbind}[4]{#1\,#3\,{\mathsf{to}}\,#2.\,#4}
% \newcommand{\qsplits}[4]{#1\,#2\,{\tsplit}\,#3.\,#4}
% \newcommand{\qforced}[4]{#1\,#2\,{\tforced}\,#3.\,#4}
% \newcommand{\qcase}[3]{#1\,#2\,{\tcases}\,\{#3\}}
% \newcommand{\qcaser}[3]{#1\,#2\,{\tcases}\,#3}
% \newcommand{\gqlet}[4]{\graybox{#1}\,#3\,{\mathsf{be}}\,#2.\,#4}
% \newcommand{\gqbind}[4]{\graybox{#1}\,#3\,{\mathsf{to}}\,#2.\,#4}
% \newcommand{\gqsplits}[4]{\graybox{#1}\,#2\,{\tsplit}\,#3.\,#4}
% \newcommand{\gqforced}[4]{\graybox{#1}\,#2\,{\tforced}\,#3.\,#4}
% \newcommand{\gqcase}[3]{\graybox{#1}\,#2\,{\tcases}\,\{#3\}}
% \newcommand{\gqcaser}[3]{\graybox{#1}\,#2\,{\tcases}\,#3}
% \newcommand{\split}[3]{\ptup #1\,{\leftarrow}\,#2;\,#3}
\newcommand{\lam}[2]{\lambda#1.\,#2}
\newcommand{\app}[2]{#1\,#2}
\newcommand{\trecord}{\mathsf{record}}
\newcommand{\recrd}[2]{\trecord_{#1}\,#2}
\newcommand{\record}[2]{\trecord_{#1}\{#2\}}
\newcommand{\tproj}{\mathsf{proj}}
\newcommand{\proj}[2]{\tproj_{#1}\,{#2}}

% Judgements
\newcommand{\SubstTy}[4]{#1#2 \vdash #3 : #4}
\newcommand{\pSubstTy}[4]{(#1)#2 \vdash (#3) : (#4)}
\newcommand{\ValTy}[3]{#1 \vdash #2 : #3}
\newcommand{\oCompTy}{\ValTy}
\newcommand{\CompTy}[4]{#1 \vdash #2 : #4 \mid #3}
\newcommand{\gpValTy}[4]{\graybox{(#1)}#2 \vdash #3 : #4}
\newcommand{\gqValTy}[4]{\graybox{#1}#2 \vdash #3 : #4}
\newcommand{\gpCompTy}[4]{\graybox{(#1)}#2 \vdash #3 : #4}
\newcommand{\gqCompTy}[4]{\graybox{#1}#2 \vdash #3 : #4}
\newcommand{\pValTy}[4]{(#1)#2 \vdash #3 : #4}
\newcommand{\qValTy}[4]{#1#2 \vdash #3 : #4}
\newcommand{\pCompTy}[4]{(#1)#2 \vdash #3 : #4}
\newcommand{\qCompTy}[4]{#1#2 \vdash #3 : #4}
%\newcommand{\CompTy}[4]{#1 \vdash #2 \mathrel{{:}\langle #3 \rangle} #4}
\newcommand{\ru}{\dfrac}
\newcommand{\nru}[3]{#1\;\dfrac{#2}{#3}}
\newcommand{\rux}[3]{\dfrac{#1}{#2}\;#3}
\newcommand{\nrux}[4]{#1\;\dfrac{#2}{#3}\;#4}

\newcommand{\rulename}[1]{\ensuremath{\mbox{\textsc{#1}}}\xspace}
\newcommand{\rbeta}[1]{\ensuremath{\beta\mbox{-}\mathord{#1}}\xspace}
\newcommand{\reta}[1]{\ensuremath{\eta\mbox{-}\mathord{#1}}\xspace}
\newcommand{\rpi}[1]{\ensuremath{\pi\mbox{-}\mathord{#1}}\xspace}
\newcommand{\rintro}[1]{\ensuremath{\mathord{#1}\mbox{-\rulename{intro}}}\xspace}
\newcommand{\relim}[1]{\ensuremath{\mathord{#1}\mbox{-\rulename{elim}}}\xspace}
\newcommand{\rvar}{\rulename{var}}
\newcommand{\rlet}{\rulename{let}}
\newcommand{\rsub}{\rulename{sub}}
\newcommand{\rweak}{\rulename{weak}}

% Concrete effects
\newcommand{\tprint}{\mathsf{print}}
\newcommand{\print}[2]{\tprint\,#1.\,#2}
\newcommand{\toutput}{\mathsf{output}}
%\newcommand{\output}[2][]{\toutput_{#1}\,#2}
\newcommand{\tthrow}{\mathsf{throw}}
\newcommand{\throw}[1]{\tthrow\,#1}
\newcommand{\tcatch}{\mathsf{catch}}
\newcommand{\catch}[3]{#1\,\tcatch\,#2 \mapsto #3}
\newcommand{\traise}{\mathsf{raise}}
%\newcommand{\raise}[1]{\traise\,#1} %clash
\newcommand{\thandle}{\mathsf{handle}}
\newcommand{\handle}[4][]{\thandle_{#1}\,#2\,#3\,#4}


\section{Introduction}

Levy's call-by-push-value calculus (CBPV) \citeyearpar{levy:hosc06}
is a simply typed lambda
calculus with disjoint sums and eager and lazy products that allows
the modelling of ``lazy'' effects.  For instance, the CBPV-program
\[\print {\mathtt{"function"}} \lam x {\print {\mathtt{"argument"}} x}\]
of type $P \To \diamond P$, will, unlike the corresponding ML-program,
not print the word ``function'' and return a function; it will simply
wait for an input value of type $P$.  Only after an argument has been
supplied, the program will print the words ``function'' and
``argument'' and then return the argument.  The reason is that effects
in CBPV can only be observed at \emph{value types} like $P$, and
function types are not value types but computation types.  Effectful
actions (like printing) at computation types $N$ are ``pushed down''
the type structure until they reach a position at a monadic type
$\diamond P$ where they can be executed.  Semantically, this is
facilitated by interpreting computation types as monad algebras with
an action $\trun : \T\,N \to N$ that allows to formally run effects
transported by the monad $\T$ at type $N$, even though $N$ is not
directly a monadic type $\diamond P$.

CBPV polarizes types into value types $P$ and computation types $N$
where the former embed into the latter as $\diamond P$ by the formal
``monad''~$\diamond$ and the latter embed into the former as $\Box N$
by ``thunking''~$\Box$.  In Levy's \emph{behavioral semantics},
functor $\diamond$ is a left adjoint to functor $\Box$ and need not be
a monad.  Quite the opposite, when modelling store
$\diamond P = P \times S$ is a comonad and $\Box N = S \to N$ a monad.
Contrastingly, in the above explained \emph{algebra semantics},
$\diamond$ is indeed modeled by a monad $\T$, yet $\Box$ just by the
identity.  It is however possible to model $\Box$ by a monoidal
comonad $\D$, and positive types by comonad coalgebras with an action
$\texpose : P \to \D\,P$ that makes the services of the comonad
available at all value types, not just at $\Box N$.

Building on these observations, we investigate in this article how
CBPV %would look like if it would
can
keep track of not just the presence
of an effect or coeffect but also the kind of effect and coeffect.
Semantically, precise information about effects can be obtained by
using a family of monads $\T[e]$ graded over effect classifiers $e$
drawn from a preordered monoid \cite{katsumata:popl14}.  For CBPV, we
generalize the concept of monad algebras $N$ to \emph{graded} such
ones, $N_e$.  Syntactically, computation typing $\CompTy \Gamma t e N$
is extended by an effect classifier $e$
(as usual in type and effect systems \citep{nielsen:effect99})
such that a computation can be
interpreted by a morphism $\Gamma \to N_e$.

Coeffect type systems
\citep{brunel:esop14,orchard:icfp14,ghicaSmith:esop14} have been
developed to track resource consumption by attaching usage information
to each variable in the typing context.  Coeffect typing generalizes
linear typing \citep{girard:linear} to quantitative typing
\citep{sergeyVytiniotisPeytonJones:popl14,mcBride:wadler60,atkey:lics18}
and subsumes sensitivity analysis \citep{reedPierce:icfp10} and static
information control flow \citep{volpano:jcs96} aka security typing
\citep{abadiBanerjeeHeintzeRiecke:popl99}.
Coeffects can be modeled by a comonad $\D[r]$ graded over resource
qualifiers $r$ drawn from a preordered semiring.  For the sake of
coeffect-graded CBPV, we generalize graded comonads to graded comonad
coalgebras.  Value types $P$ and contexts $\Gamma$ are then
interpreted as such coalgebras.  Value typing
$\qValTy \gamma \Gamma v P$ and computation typing
$\qCompTy \gamma \Gamma t N$ are based on linear typing an come
equipped with a resource context~$\gamma$.  Semantically, values are
interpreted as morphisms $\Gamma_{r\gamma} \to P_r$ allowing the
``multiplication'' or \emph{scaling} of a value by $r$ % natural in $r$
and computations as morphisms $\Gamma_\gamma \to N$.

Finally, effect and coeffect graded CBPV can be combined into a fully
graded CBPV calculus, surprisingly without sorting out any interaction
between effects and coeffects, such as the distributive laws of
\citet{orchard:icfp16}.  We credit the smoothness of the integration
to the careful placement of monad and comonad in CBPV's type system,
and the lack of need for scaling % of not just values but also
of computations.

\paragraph*{Contributions}
\begin{enumerate}

\item We introduce graded monad algebras and an effect-graded version
  of CBPV in Section~\ref{sec:effect}, after recapitulating monads and
  their algebras and graded monad.

\item We further introduce graded comonad coalgebras and a
  coeffect-graded version of CBPV in Section~\ref{sec:coeff}.  We give
  its denotational and operational semantics and adapt Atkey and
  Wood's substitution theorem
  \citeyearpar{atkeyWood:types19,woodAtkey:linearity20} to this
  version of CBPV.

\item We present an effect- and coeffect-graded version of CBPV in Section~\ref{sec:full-cbpv}.

\end{enumerate}

\paragraph*{Preliminaries}  The reader should bring some elementary
knowledge of category theory, such as the interpretation of
simply-typed lambda calculus in cartesian-closed categories.  However,
we try to be gentle with categorical concepts such as monads and
recapitulate their definition where needed.  In many cases, it is
sufficient to think in terms of the category $\SET$ where objects are
sets and morphisms functions between sets, or in the functor category
$[\CC \to \SET]$ where objects are monotone families of sets
$(A_i)_{i:\CC}$ indexed by objects $i$ of $\CC$ such that
$\CC_i \to \CC_j$ in $\SET$ for $i \to j$ in $\CC$ and morphisms are
natural transformations $(f_i : A_i \to B_i)_{i:\CC}$.

\section{An effect-graded version of CBPV}
\label{sec:effect}

% As effect algebra, we assume a pre-ordered monoid
% $(\Eff,\bu,\Ge,\leq)$ such that $\_{\bu}\_$ is monotone wrt.\ the
% preorder $\leq$ in both arguments.  The unit $\Ge$ shall mean \emph{no
%   effect} and the operation $\_{\bu}\_$ serves to accumulate effects,
% possibly in a sequential order---unless the monoid is commutative.  The
% preorder represents effect subsumption, i.e., loss in precision of the
% effect analysis.  Note that the unit $\Ge$ is not necessarily the least element
% wrt.\ $\leq$.

\subsection{Recapitulation: modelling effects via monads}
\label{sec:ccc}

In this section, we recapitulate monads and some essential vocabulary
of category theory.  There are no surprises, thus, the experienced
reader is invited to skip this section.

Consider a categorical model $\CC$ of the simply-typed lambda calculus
(STLC), i.e., where types $\tau$ and contexts $\GG$ are interpreted as
objects $\den \tau$ and $\den \Gamma$ of $\CC$ and terms
$\GG \vdash t : \tau$ as morphisms $\dent t \in \CC(\den\GG,\den\tau)$.
Such a category could be $\SET$, interpreting types as sets and terms
as functions, mapping the valuation of their free variables to their
value, or $\CPO$, interpreting types as complete partial orders and
terms as monotone functions, or a presheaf model of the STLC etc.
Typically, $\CC$ is a cartesian-closed category, i.e., has products
$A_1 \times A_2$ of objects to model product types and contexts,
and exponentials $A \To B$ to model function types.
Further $\CC$ maybe be distributive,
\ie, have coproducts $A_1 + A_2$ that distribute over products, to
model variants aka disjoint sum types.

Recall that effects are modeled by a suitable monad $\T : \CC \to \CC$
in $\CC$.  We shall refer to elements of a monadic type $\T\,A$ as
\emph{computations} when $\CC = \SET$.  For a general $\CC$,
computations shall be the morphisms $\Hom A {\T\,B}$ called
\emph{Kleisli arrows}.
An example module would be the \emph{writer monad}
$\T\,A = \String \times A$
that models the effect \emph{output}.
Another example would be the \emph{exception monad} $\Exc + \_$ where
the result of a computation is either an exception $e : \Exc$ or a
regular result.
The monad operations, together with
their implementation for \emph{writer} in $\SET$, are the following:
\[
\begin{array}{lcl}
\fmap[\T] & : & (A \to B) \to \T\,A \to \T\,B \\
\fmap[\T] f\, (s,\; a) & = & (s,\, f\, b)
\\[1ex]
\return[\T] & : & A \to \T\,A \\
\return[\T]\, a & = & (\verb|""|, a)
\\[1ex]
\join[\T] & : & \T\,(\T\,A) \to \T\,A \\
\join[\T]\,(s_1,(s_2,a)) & = & (s_1 \append s_2,\; a) \\
\end{array}
\]
Herein, \verb|""| shall denote the empty string and $\append$ string
concatenation.

Monad unit $\treturn$ (written $\eta$ in category speak) turns a value
into a pure computation and monad multiplication $\tjoin$ (written
$\mu$ in category speak) combines two effects and thus allows
sequencing of computations via \emph{Kleisli composition}
$ (g : \Hom B {\T\,C}) \comp_{\T} (f : \Hom A {\T\,B}) = (\tjoin \comp
  \fmap g \comp f : \Hom A {\T\,C})$.

The presence of $\tfmap : \Hom A B \to \Hom {\T\,A} {\T\,B}$
satisfying the functor laws $\tfmap\,\tid = \tid$ and
$\tfmap\,(f \circ g) = \tfmap\,f \circ \tfmap\,g$ makes $\T$ a
\emph{(endo)functor}, written $\T : [\CC \to \CC]$.  The
endofunctors $F : [\CC \to \CC]$ form a category, the
$\emph{functor category}$, with identity $\Id\,A = A$ and
composition $(F \comp G)\,A = F\,(G\,A)$.  We shall write
$f : F \todot G$ for morphisms in the functor category, called
\emph{natural transformations}.  These are families
$f_A : \Hom {F\,A} {G\,A}$ of morphisms that commute with the
functor action $\tfmap$, \ie,
$\fmap[G]h \comp f_A = f_B \comp \fmap[F]h$ for any $h : \Hom A B$.

Unit $\treturn : \Id \todot \T$ and multiplication
$\tjoin : \T \comp \T \todot \T$ are \emph{natural transformations}
which breaks down to $\fmap[\T] h \comp \treturn = \treturn \comp h$
and $\fmap[\T] h \comp \tjoin = \tjoin \comp \fmap{(\fmap h)}$.
%
The three equational
laws of $\tjoin$ can be visualized compactly in the following
commutative diagram.
\[
\xymatrix@C=12ex{
  \T\,A     \ar[r]^{\treturn} \ar@{=}[dr]^{\tid}
& \T\,(\T\,A) \ar[d]^{\tjoin}
& \T\,(\T\,(\T\,A)) \ar[l]_{\tjoin} \ar[d]^{\tfmap\,\tjoin}
\\
& \T\,A       \ar@{=}[dr]^{\tid}
& \T\,(\T\,A) \ar[l]_{\tjoin}
\\
&
& \T\,A \ar[u]_{\tfmap\,\treturn}
}
\]

\subsection{Effect algebras and graded monads}
\label{sec:graded_monad}

Following \citet{katsumata:popl14}
we can obtain more information about effects using a monad $\T[e]$
\emph{graded} over elements $e$ of a suitable effect algebra $\Eff$.
An effect algebra be a preordered monoid
$(\Eff,\bu,\Ge,\leq)$ such that $\_{\bu}\_$ is monotone wrt.\ the
preorder $\leq$ in both arguments.  The unit $\Ge$ shall mean \emph{no
  effect} and the operation $\_{\bu}\_$ serves to accumulate effects,
possibly in a sequential order---unless the monoid is commutative.  The
preorder represents effect subsumption, i.e., loss in precision of the
effect analysis.  Note that the unit $\Ge$ is not necessarily the least element
wrt.\ $\leq$.

Say we want to track an upper bound on the length of the output
produced by a program.  To this end, we can use the preordered monoid
$\Eff = \bN \cup \{\infty\}$ under addition $\_{\bu}\_ = \_{+}\_$
with unit $\Ge = 0$ and the
natural order $\leq$.  The effect $\infty$ then denotes unbounded output, or
output whose length we cannot track in the type system (\eg, when it
depends on some variable).  Increasing the upper bound along $\leq$
means loss of precision of our analysis, with $\infty$ the least
precise information, meaning no upper bound.  The corresponding graded
writer monad is $\T[e]\,A = (\StringLe e) \times A$ where the output is an
element of $\StringLe e$, a string of length at most $e \in \bN \cup
\{\infty\}$.  The operations of a graded monad are, again given with
their $\SET$-implementation for the graded writer:
\[
\begin{array}{lcl}
\tfmap & : & (A \to B) \to \T[e]\,A \to \T[e]\,B \\
\tfmap\, f\, (s,\; a) & = & (s,\, f\, b)
\\[1ex]
\treturn & : & A \to \T[\Ge]\,A \\
\treturn\, a & = & (\verb|""|, a)
\\[1ex]
\tjoin & : & \T[e_1]\,(\T[e_2]\,A) \to \T[e_1 \bu e_2]\,A \\
\tjoin\,(s_1,(s_2,a)) & = & (s_1 \append s_2,\; a)
\\[1ex]
\tcast & : & \T[e]\,A \to \T[e']\,A \mbox{ for } e \leq e'\\
\tcast & = & \tid \\
\end{array}
\]
A graded version of the exception monad would use effect algebra
$\Eff = \Pot\Exc$ under union and subset; an effect $e$ is a set of
possible exceptions thrown by a computation.\footnote{An example of a
  exception-tracing type system is the Java language.}

The interesting laws for graded monads are given by the following
commutative diagrams.
\[
\xymatrix@C=8ex{
  \T[e]\,A     \ar[r]^{\treturn} \ar@{=}[dr]^{\tid}
& \T[\Ge]\,(\T[e]\,A) \ar[d]^{\tjoin}
\\
& \T[e]\,A
}
\quad
\xymatrix@C=10ex{
  \T[e]\,A     \ar[r]^{\tfmap\,\treturn} \ar@{=}[dr]^{\tid}
& \T[e]\,(\T[\Ge]\,A) \ar[d]^{\tjoin}
\\
& \T[e]\,A
}
\quad
\xymatrix@C=8ex{
  \T[e_1 \bu e_2]\,(\T[e_3]\,A) \ar[d]^{\tjoin}
& \T[e_1]\,(\T[e_2]\,(\T[e_3]\,A)) \ar[l]_{\tjoin} \ar[d]^{\tfmap\,\tjoin}
\\
  \T[e_1 \bu e_2 \bu e_3]\,A
& \T[e_1]\,(\T[e_2 \bu e_3]\,A) \ar[l]_{\tjoin}
}
\]
Further, $\tcast$ commutes with $\tfmap$, namely $\tfmap\,f \circ
\tcast = \tcast \circ \tfmap\,f$, and in two ways with $\tjoin$, namely
$\tjoin \circ \tcast = \tcast \circ \tjoin$ and $\tjoin
\circ \tfmap\,\tcast = \tcast \circ \tjoin$.
We may write $\T[e_1 \leq e_2]$ for the natural transformation
$\tcast : \T[e_1] \todot \T[e_2]$.

\begin{remark}[$\Eff$-graded monad are lax monoidal functors $\T : {[\Eff \to [\CC \to \CC]]}$]
The concept of a graded monad can be more succinctly expressed by
using more advanced language of category theory.  (This reformulation
is not essential for the remainder of the exposition and may be
skipped on first reading.)

Recall that a \emph{monoidal category} $\EE$ has a designated object
$\I : \EE$ and an operation $\_{\otimes}\_ : [\EE \to [\EE \to \EE]$,
the \emph{tensor product} on objects of $\EE$ that is functorial in
both positions.  Further, there are natural isomorphisms
$\Gl : (\I \otimes A) \cong A$ and $\Gr : (A \otimes \I) \cong A$
witnessing the unitality of $\I$ and
$\alpha : (A \otimes (B \otimes C)) \cong ((A \otimes B) \otimes C)$
witnessing associativity of $\otimes$.

Any preordered monoid, such as $(\Eff,\_\bu\_,\Ge,\leq)$, makes a
monoidal category $\EE = \Eff$ with Homset
$\EE(e,e') = \{ () \mid e \leq e' \}$, tensor $\otimes = \bu$ and unit
$\I = \Ge$.  Further, this category is \emph{thin}, \ie, there is at
most one morphism between any two objects $e,e'$.

A graded monad $\T : [\Eff \to [\CC \to \CC]]$ is then a morphism from
monoidal category $(\Eff,\bu,\Ge)$ to monoidal category
$([\CC \to \CC],\comp,\Id)$.  The monoidal structure in the latter is
just functor composition and identity functor.  Operations $\tcast$
and $\tfmap$ witness the functoriality of $\T$ in its first and second
argument.  The operations $\treturn$ and $\tjoin$ witness (in a
directed way) the preservation of unit $\Ge$ and multiplication $\bu$
by $\T$, making $\T$ a \emph{lax monoidal functor}.
(End of remark.)
\end{remark}

\subsection{CBPV, monad algebras, and their graded version}
\label{sec:graded_algebra}

Call-by-push value \cite{levy:hosc06} is a refinement of Moggi's
computational lambda-calculus \cite{moggi:infcomp91} that allows effects
not only in monadic types, \ie, in objects $\T\,A$, but more generally
in computation types.  These correspond to monad algebras for $\T$ in $\CC$,
aka $\T$-algebras.
Those algebras are objects $B$ together with a morphism $\run[B] : \T\,B
\to B$ that allows to formally \emph{run} the monad, ``merging'' its
effects into $B$.
The prime example of a monad algebra is simply a monadic
type, because $\run[\T\,A] : \T\,(\T\,A) \to \T\,A$ is just
$\tjoin$.  Levy \citeyearpar{levy:hosc06} shows that monad algebras are
closed under products ($\times$) and exponentiation ($\To$)
with arbitrary objects.  \Eg,
in $\SET$ we can define, for \emph{writer} algebras:
\[
\begin{array}{lcl}
%   \run[B_1 \times B_2] & : & \T\,(B_1 \times B_2) \to B_1 \times B_2 \\
%   \run[B_1 \times B_2]\, (s,(b_1,b_2)) & = & (\run[B_1] (s,b_1),\
%                                            \run[B_2] (s,b_2))
% \\[1ex]
  \run[B \times B'] & : & \T\,(B \times B') \to B \times B' \\
  \run[B \times B']\, (s,(b,b')) & = & (\run[B] (s,b),\
                                           \run[B'] (s,b'))
\\[1ex]
  \run[A \To B] & : & \T\,(A \To B) \to (A \To B) \\
  \run[A \To B]\, (s,f) & = & \lambda a.\; \run[B]\,(s,\,f\,a)
\\
\end{array}
\]
These definition implement \emph{lazy} effects that cannot be observed
at computation types such as $A \To B$, but only at value types; the
$\run[]$ of the monad algebra pushes the effects towards result types
that are eventually types of observable objects (values).

With $\run[]$ being a generalization of $\tjoin$, the laws for $\run[]$
are in analogy of those for $\tjoin$ (if $B$ were $\T\,A$):
\[
\xymatrix@C=10ex{
  B     \ar[r]^{\treturn} \ar@{=}[dr]^{\tid}
& \T\,B \ar[d]^{\run[]}
& \T\,(\T\,B) \ar[l]_{\tjoin} \ar[d]^{\tfmap\,\run[]}
\\
& B
& \T\,B \ar[l]_{\run[]}
}
\]

In graded CBPV, monad algebras get replaced by \emph{graded monad
  algebras}.  Given a graded monad $\T$, a $\T$-algebra is a family of
objects $\left(B_e\right)_{e:\Eff}$ and morphisms
$\run[B] : \T[e_1]\,B_{e_2} \to B_{e_1 \bu e_2}$ satisfying these
laws:
\[
\xymatrix@C=10ex{
  B_e     \ar[r]^{\treturn} \ar@{=}[dr]^{\tid}
& \T[\Ge]\,B_e \ar[d]^{\run[]}
\\
& B_e
}
\qquad
\xymatrix@C=10ex{
  \T[e_1 \bu e_2]\,B_{e_3} \ar[d]^{\run[]}
& \T[e_1]\,(\T[e_2]\,B_{e_3}) \ar[l]_{\tjoin} \ar[d]^{\tfmap\,\run[]}
\\
  B_{e_1 \bu e_2 \bu e_3}
& \T[e_1]\,B_{e_2 \bu e_3} \ar[l]_{\run[]}
}
\]
Further the family $B$ should be functorial in the sense that there is
a family of coercion morphisms $B_{e_1 \leq e_2} : B_{e_1} \to B_{e_2}$ with
$B_{e \leq e} = \tid$ and $B_{e_2 \leq e_3} \circ B_{e_1 \leq e_2} =
B_{e_1 \leq e_3}$.

Graded monad algebras are as well closed under
pointwise products $(B \times B')_e = B_e \times B'_e$ and
exponentiation with objects $(A \To B)_e = A \to B_e$;
here the graded writer example:
\[
\begin{array}{lcl}
  \run[B \times B'] & : & \T[e_1]\,(B \times B')_{e_2} \to (B \times
                          B')_{e_1 \bu e_2} \\
  \run[B \times B']\, (s,(b,b')) & = & (\run[B] (s,b),\
                                           \run[B'] (s,b'))
\\[1ex]
  \run[A \To B] & : & \T[e_1]\,(A \To B)_{e_2} \to (A \To B)_{e_1 \bu e_2} \\
  % \run[A \To B] & : & \T[e_1]\,(A \To B_{e_2}) \to (A \To B_{e_1 \bu e_2}) \\
  \run[A \To B]\, (s,f) & = & \lambda a.\; \run[B]\,(s,\,f\,a)
\\
\end{array}
\]

\begin{remark}[Naturality of $\trun$]
  The $\T$-algebra $B : [\Eff \to \CC]$ comes with a natural morphism
  $\run[e_1,e_2] : \Hom {\T[e_1]\,B_{e_2}} {B_{e_1 \bu e_2}}$ in the
  sense that
  $B_{(e_1\leq e_1') \bu (e_2 \leq e_2')} \comp {\run[e_1,e_2]} =
  {\run[e_1',e_2']} \comp \T[e_1 \leq e_1']\,B_{e_2 \leq e_2'}$.
\end{remark}


\begin{figure}[htbp]
\flushleft
\ruler{}
Types.
\[
\begin{array}{lllrl@{\qquad}l}
\PTy & \ni & P
  & ::= & \graybox{\thunkty e N}
     \mid o \mid \sumty i I P \mid \tupty i I P
  & \mbox{Value types (positive types)} \\
\NTy & \ni & N
  & ::= & \diamond P
     \mid P \To N \mid \recty i I N
  & \mbox{Computation types (negative types)} \\
\Cxt & \ni & \Gamma
  & ::= & \cempty \mid \ext \Gamma x P
  & \mbox{Typing context}
\\
\end{array}
\]
\dashruler{}
Terms.
\[
\begin{array}{lllrl@{\qquad}l}
\PTm & \ni & v,w
  & ::= & x
     \mid \thunk t
     \mid \inj i v
     \mid \ptup{\bar v}
  & \mbox{Values (positive terms)} \\
\NTm & \ni & t,u
  & ::= &
         \vlet x v t
  & \mbox{Computations (negative terms):}
\\ &&& \mid &
        \force v
    \mid \case v {\overline{x.\,t}}
    \mid \splits v {\bar x} t
  & \mbox{value eliminations}
\\ &&& \mid &
         \ret v        \mid \bind x u t
  & \mbox{monad operations}
\\ &&& \mid &
         \lam x t      \mid \app t v
  & \mbox{functions}
\\ &&& \mid &
         \trecord\{\overline{i:t}\} % \ntup{\bar t}
       \mid \proj i t
  & \mbox{lazy tuples (records)}
\end{array}
\]
\dashruler{}
Value typing \fbox{$\ValTy \Gamma v P$}\,.
\begin{gather*}
 \nru{\rvar}
     {x{:}P \in \Gamma
    }{\ValTy \Gamma x P}
\qquad
 \nru{\rintro\Box}
     {\CompTy \Gamma t {\graybox e} N
    }{\ValTy \Gamma {\thunk t} {\thunkty {\graybox e} N}}
\\[1.5ex]
 \nru{\rintro\GS}
     {\ValTy \Gamma v {P_i}
    }{\ValTy \Gamma {\inj i v} {\sumt I P}}
\qquad
 \nru{\rintro\otimes}
     {\forall i{:}I, \ \ValTy \Gamma {v_i} {P_i}
    }{\ValTy \Gamma {\ptup v} {\tupt I P}}
    % }{\ValTy \Gamma {\ptup{\overline{v_i}^{i:I}}} {\tupt I P}}
%    }{\ValTy \Gamma {\ptup (v_i)_{i:I}} {\tupt I P}}
\end{gather*}
\dashruler{}
Computation typing \fbox{$\CompTy \Gamma t {\graybox{e}} N$}\,.
\begin{gather*}
 \nru{\rlet}
     {\ValTy \Gamma v P \qquad
      \CompTy {\ext \Gamma x P} t e N
    }{\CompTy \Gamma {\vlet x v t} e N}
\qquad
 \nru{\relim\Box}
     {\ValTy \Gamma v {\thunkty {\graybox e} N}
    }{\CompTy \Gamma {\force v} {\graybox e} N}
\\[1.5ex]
 \nru{\relim\GS}
     {\ValTy \Gamma v {\sumt I P} \qquad
      \forall i{:}I,\ \CompTy {\ext \Gamma {x_i} {P_i}} {t_i} e N
    }{\CompTy \Gamma {\caser v {\{x_i.\, t_i\}_{i:I}}} e N}
\qquad
 \nru{\relim\otimes}
     {\ValTy \Gamma v {\tupt I P} \qquad
      \CompTy {\extr \Gamma {\overline{x_i{:}P_i}^{i:I}}} t e N
    }{\CompTy \Gamma {\splits v {\bar x} t} e N}
\\[1.5ex]
 \nru{\rintro\diamond}
     {\ValTy \Gamma v P
    }{\CompTy \Gamma {\ret v} {\graybox \Ge} {\diamond P}}
\qquad
 \nru{\relim\diamond}
     {\CompTy \Gamma u {\graybox{e_1}} {\diamond P} \qquad
      \CompTy {\ext \Gamma x P} t {\graybox{e_2}} N
    }{\CompTy \Gamma {\bind x u t} {\graybox{e_1 \bu e_2}} N}
\\[1.5ex]
 \nru{\rintro\To}
     {\CompTy {\ext \Gamma x P} t e N
    }{\CompTy \Gamma {\lam x t} e {P \To N}}
\qquad
 \nru{\relim\To}
     {\CompTy \Gamma t e {P \To N} \qquad
      \ValTy \Gamma v P
    }{\CompTy \Gamma {\app t v} e N}
\\[1.5ex]
 \nru{\rintro\Pi}
     {\forall i{:}I,\ \CompTy \Gamma {t_i} e {N_i}
    }{\CompTy \Gamma {\trecord {\{i : t_i\}_{i:I}}} e {\rect I N}}
    % }{\CompTy \Gamma {\ntup{\overline{t_i}^{i:I}}} e {\rect I N}}
\qquad
 \nru{\relim\Pi}
     {\CompTy \Gamma t e {\rect I N}
    }{\CompTy \Gamma {\proj i t} e  {N_i}}
\\[1.5ex]
\graybox{
 \nrux{\rsub}
      {\CompTy \Gamma t e N}
      {\CompTy \Gamma t {e'} N}
      {e \leq e'}
}
\end{gather*}
\rule{\textwidth}{0.2pt}
  \caption{Effect-graded call-by-push-value.}
  \label{fig:cbpv}
\end{figure}

\subsection{Effect-graded CBPV: syntax and typing}
\label{sec:effect-cbpv}

With the theory of graded monads in place,
we design a graded version of CBPV.
The syntax and typing rules for effect-graded CBPV are given in
Fig.~\ref{fig:cbpv}.  The differences to pure CBPV are in
\graybox{\mbox{gray boxes}}.

Types are classified into value types $P \in \PTy$ (written $A$ in
\citet{levy:hosc06}) and computation types $N \in \NTy$
(written $\underline B$ in
\loccit).  These are positive and negative types in the terminology of
focusing \cite{zeilberger:PhD}.
Positive types are generated from base types $o$ via
disjoint sums $\sumty i I P$ with tag set $I$, eager products $\tupty
i I P$ of arity $I$ (composed
from $1$ and $A_1 \times A_2$ in \loccit) and thunking $\thunkty e N$
(written $U\,\underline B$ in \loccit).  In contrast to pure CBPV,
thunk types $\thunkty e N$ are annotated with an effect $e$ that can
be triggered when the thunk is forced.  Negative types are just as in
pure CBPV: monadic types $\diamond P$ (written $F\,A$ in \loccit),
function types $P \To N$ (written $A \to \underline B$) in \loccit)
and record types $\recty i I N$ with label set $I$.
Records are \emph{lazy} tuples whose components are only computed by demand.
We abbreviate $\sumty i I P$ by its ``meta-level $\eta$-contraction''
$\sumt I P$; notations $\tupt I P$ and
$\rect I N$ are understood analogously.

Terms are separated into values $v \in \PTm$ and computations $t \in
\NTm$ and identical to pure CBPV, modulo changes in the concrete
syntax.  Values introduce positive types,
computations introduce
negative types and
eliminate both positive and negative types.
We use bars to indicate sequences, e.g., $\bar v$ for a
sequence of values, but drop the bar when the context of discourse
makes clear that we are dealing with sequences rather than single
objects.  For instance, ``$v_i$'' (where $i : I$) in the premise of
\rintro\otimes indicates that $v$ is a sequence of values with
elements $(v_i)_{i:I}$.  We may abbreviate $\trecord \{i :
t_i\}_{i:I}$ by $\recrd I t$ where $I$ is the label set and $t$ a
mapping from labels $i:I$ to terms $t_i$.

The meaning of the term constructors is best understood via their
typing.  Typing contexts $\Gamma$ are finite maps from variables $x$
to value types $P$, with $\ext \Gamma x P$ denoting the update of the
finite map $\Gamma$ at key $x$ with value $P$.

Value typing $\ValTy \Gamma v P$ is just as in pure CBPV, however,
computation typing $\CompTy \Gamma t e N$ also records effects $e :
\Eff$ potentially produced at runtime by computation $t$.  Thunking a
computation (rule \rintro\Box) stores the inferred effect classifier
$e$ in the thunk type $\thunkty e N$.

Effects are accumulated via the introduction and elimination rule for
the graded monad.  The unit $\ret v$ of the monad is effect-free
(\rintro\diamond); running this computation just produces the pure
value $v$.  Sequencing computations $u$ and $t$ via the \emph{bind}
construct ``$\bind x u t$'' composes the effects $e_1$ of $u$ with the
effects $e_2$ of $t$ in that order.  The intuition is that first $u$
is run, producing effects classified by $e_1$, and its result is bound
to $x$ to run $t$, producing effects classified by $e_2$.  The sum of
the effects is classified by $e_1 \bu e_2$.

The other introduction and elimination rules are just as in pure
CBPV, except that they propagate the effect classifier $e$ from
hypothesis to conclusion.  Note that in case distinction (\relim\GS)
and record construction (\rintro\Pi) all subterms $t_i$ are required
to produce effects classified by the same $e$.  However, in reality,
different branches of \eg a case distinction may produce
very different effects.  To end up with a unique classifier $e$, the
branches may have to be typed using effect subsumption (\rsub).  In
fact, the uses of \rsub can be confined to the hypotheses of \relim\GS
and \rintro\Pi, except for a final invocation of \rsub at the very end
of the typing derivation.  Alternatively, we could have introduced
effect algebras with suprema $\sup_{i:I} e_i$ instead of a preorder $e
\leq e'$.  However, suprema might not always exist; by using
subsumption \rsub, we delegate the problem of partiality to the
construction of a typing derivation.

\begin{remark}\label{rem:cbpv}
  We recover pure CBPV from graded CBPV using the trivial effect
  algebra $\Eff = \{\Ge\}$.
\end{remark}

\subsection{Effect graded CBPV: denotational semantics}

The denotational semantics of the novel parts
of graded CBPV has been informally explained in
sections \ref{sec:ccc} to \ref{sec:graded_algebra} already; in the following,
we spell out the details.  We assume a distributive cartesian-closed
category $\CC$ with a strong graded monad $\T : [\Eff \to [\CC \to \CC]]$.
Let us agree on some notation for the constructions on objects and
morphisms:
\begin{itemize}
\item Product $\prod_{i:I} A_i$ with projections $\pi_i : \CC(\prod_I
  A, A_i)$ and tupling $\langle f_i
  \rangle_{i:I} : \CC(C,\prod_i A)$ for $f_i : \CC(C,A_i)$.  Binary
  products $\prod_{\{1,2\}} A$ are written $A_1 \times A_2$,
  and the nullary product (terminal object)
  is written $1$ with nullary tupling $\langle\rangle : \CC(C,1)$.
\item Coproduct $\coprod_{i:I} A_i$ with injections $\iota_i :
  \CC(A_i,\coprod_I A)$ and cotupling $[f_i]_{i:I} : \CC(\coprod_I A,
  B)$ for $f_i : \CC(A_i,B)$.
\item Exponential $A \To B$ with $\Lambda : \CC(C \times A, B) \to
  \CC(C, A \To B)$ and $\teval : \CC((A \To B) \times A, B)$.
\item Graded monad $\T$ with functoriality $\fmap[\T] f :
  \CC(\T[e]A,\T[e]B)$ for $f : \CC(A,B)$,
  unit $\treturn : \CC(A,\T[\Ge]A)$,
  multiplication $\tjoin : \CC(\T[e_1](\T[e_2]A),\T[e_1 \bu e_2]A)$,
  strength $\strengthl : \CC(A \times \T[e]B, \T[e](A \times B))$ and
  coercion $\tcast : \CC(\T[e]A,\T[e']A)$ for $e \leq e'$.
  Costrength $\strengthr : \CC(\T[e]A \times B, \T[e](A \times B))$
  is derivable in the standard way.
\end{itemize}

\subsubsection{Interpretation of types}

Positive types $P$ are interpreted as objects $\denp P$ of $\CC$,
where a denotation $\denp o : \CC$ of base types $o$ is assumed.
This interpretation lifts to contexts $\Gamma$ via $\denp \Gamma =
\prod_{x:\dom(\Gamma)} \denp{\Gamma(x)}$.

Negative types $N$ are interpreted
as functors
$\denn N : [\Eff \to \CC]$
%$\denn N : (\Eff,\leq) \to \CC$
mapping
effect classifiers $e$ to objects $\Denn N e$ and effect subsumption $e
\leq e'$ to morphisms $\Denn N {e \leq e'} : \Denn N e \to \Denn N {e'}$.
\[
\begin{array}{lcl}
  \denp\_ & : & \PTy \to \CC \\
  \denp{\thunkty e N} & = & \Denn N e \\
  \denp{\sumt I P} & = & \coprod_{i:I} \denp{P_i} \\
  \denp{\tupt I P} & = & \prod_{i:I} \denp{P_i} \\
\end{array}
\qquad\qquad
\begin{array}{lcl}
  \Denn\_\_ & : & \NTy \to \Eff \to \CC \\
  \Denn{\diamond P} e & = & \T[e] \denp P \\
  \Denn{P \To N}   e & = & \denp P \To \Denn N e \\
  \Denn{\rect I N} e & = & \prod_{i:I} \Denn{N_i}e \\
  % \denn{P \To N}   e & = & \denp P \To (\denn N e) \\
  % \denn{\rect I N} e & = & \prod_{i:I} \left(\denn{N_i}e\right) \\
\end{array}
\]
The graded $\T$-algebra structure $\run[\denn N]$ is constructed by
induction on $N$, as well as functoriality $\Denn N {e_1 \leq e_2}$:
\[
\begin{array}{l@{~~}c@{\hspace{1ex}}l}
  \run[\denn N] & : & \Hom {\T[e_1](\Denn N {e_2})} {\Denn N {e_1 \bu e_2}} \\
  \run[\denn {\diamond P}] & = & \tjoin \\
  \run[\denn {P \To N}]   & = & \Lambda(\run[\denn N] \circ
                                \fmap[\T]\teval \circ \strengthr) \\
  \run[\denn {\rect I N}] & = & \langle \run[\denn{N_i}] \comp
                                \fmap[\T]{\pi_i} \rangle_{i:I} \\
\end{array}
\
%\qquad
\begin{array}{r@{\hspace{1ex}}c@{\hspace{1ex}}l}
  \Denn N {e_1 \leq e_2} & : & \Hom {\Denn N {e_1}} {\Denn N {e_2}} \\
  \Denn {\diamond P} {e_1 \leq e_2} & = & \tcast \\
  \Denn {P \To N}   {e_1 \leq e_2} & = &
    \Lambda(\Denn N {e_1 \leq e_2} \comp \teval) \\
  \Denn {\rect I N} {e_1 \leq e_2} & = &
    \langle \Denn N {e_1 \leq e_2} \comp \pi_i {} \rangle_{i:I} \\
\end{array}
\]
This construction is the same as \citet{levy:hosc06}, modulo grading.
The algebra laws ensue.
That the coercions $\Denn N {e_1 \leq e_2}$
satisfy identity and composition---the functor
laws for $\denn N$---is easy to verify.

% The coercion morphisms $\Denn N {e_1 \leq e_2}$ are obtained
% analogously:
% \[
% \begin{array}{rcl}
%   \Denn N {e_1 \leq e_2} & : & \Hom {\Denn N {e_1}} {\Denn N {e_2}} \\
%   \Denn {\diamond P} {e_1 \leq e_2} & = & \tcast \\
%   \Denn {P \To N}   {e_1 \leq e_2} & = &
%     \Lambda(\Denn N {e_1 \leq e_2} \comp \teval) \\
%   \Denn {\rect I N} {e_1 \leq e_2} & = &
%     \langle \Denn N {e_1 \leq e_2} \comp \pi_i {} \rangle_{i:I} \\
% \end{array}
% \]
% That these coercions satisfy the identity and composition---the functor
% laws for $\denn N$---is easy to verify.

\subsubsection{Interpretation of terms}

Values $\ValTy \Gamma v P$ are interpreted as morphisms $\dent v :
\Hom {\denp \Gamma} {\denp P}$ just as in pure CBPV.  Computations
$\CompTy \Gamma t e N$ are interpreted as morphisms $\dent t : \Hom
{\denp \Gamma} {\Denn N e}$.  Most of the cases are straightforward
and in analogy to CBPV, so let us focus on the
\graybox{\mbox{modified}} rules where grading comes into play.  A
subtlety is that we interpret \emph{typing derivations} rather than terms,
because rule \rsub is a silent construction on raw terms but becomes a
coercion in the denotational semantics.  The most natural way to make
this precise is to use intrinsically well-typed syntax.  Going from typing
rules to such well-typed syntax is a routine transformation which we
do not spell out here.  The reader interested in well-typed syntax for
CBPV is referred to \citet{abelSattler:ppdp19}.
The other subtlety, that we ignore variable names in the
interpretation, can also be made precise by well-typed syntax which
uses de Bruijn indices.

As in Levy's algebra semantics \citeyearpar{levy:hosc06}, creation and
forcing of thunks is invisible in the model.  The bind operation
(\relim\diamond) utilizes the generalization of $\tjoin$ to $\trun$.
Subsumption is interpreted by functoriality in $\Eff$.
\[
\begin{array}{llcl}
\rintro\Box & \dent{\thunk t} & = & \dent t \\
\relim\Box  & \dent{\force v} & = & \dent v
\\[1ex]
\rintro\diamond & \dent{\ret v} & = & \treturn \comp \dent v \\
\relim\diamond  & \dent{\bind x u t} & = &
  \run[\denn N] \comp \fmap[\T]\teval \comp \strengthl \comp
  \langle \Lambda\dent t,\,\dent u \rangle
\\[1ex]
\rsub & \dent t & = & \Denn N {e \leq e'} \comp \dent t
\\
\end{array}
\]
These definitions are understood in the context of the given typing rules.


\subsection{Example effects}

We replay some of Levy's \citeyearpar{levy:hosc06} effect examples in
graded CBPV.

\paragraph{Printing}

Outputting a fixed string $s$ before computing $t$ is
facilitated by ``$\print s t$'' which we type by the following rule:
\[
  \rux{\CompTy \Gamma t m N
     }{\CompTy \Gamma {\print s t} {n + m} N
     }{s : \StringLe n}
\]
Herein, we use effect algebra $(\bN \cup \{\infty\},+,0,\leq)$ and
graded monad $\T[n]A = \StringLe n \times A$.
%
This allows us to implement a family of morphisms
$\toutput : \StringLe n \to \Hom[\SET] {1} {\T[n]\,1}$ to interpret
the print statement as
$\dent {\print s t} = \run[\den N] \comp \fmap[\T]{\pi_2} \comp
\strengthr \comp \langle \toutput\,s \comp \langle\rangle,\, \dent t
\rangle$:
\[
\xymatrix@C=8ex{
  \den \Gamma
     \ar[rr]^{\langle \toutput\,s \comp \langle\rangle,\, \dent t \rangle}
&& \T[n]\,1 \times \Denn N m
     \ar[r]^{\strengthr}
& \T[n](1 \times \Denn N m)
     \ar[r]^{\fmap[\T]{\pi_2}}
& \T[n]\,\Denn N m
     \ar[r]^{\trun}
& \Denn N {n + m}
}
\]

% This allows us to implement a family of natural transformations $\toutput_A : \StringLe n \to \Hom[\SET] {\T[m]\,A} {\T[n+m]\,A}$

\paragraph{Exceptions}

Given a set $\Exc$ of exceptions whose elements we refer to by $e$ and whose subsets by $E$,
consider the effect algebra $(\Pot\Exc,\cup,\emptyset,\subseteq)$.
Primitives for throwing and catching exceptions can be added to graded
CBPV by the following rules:
\newcommand{\eee}{E_1\setminus\{e\} \cup E_2}
\[
  \ru{}{\CompTy \Gamma {\throw e} {\{e\}} N} %{e : \Exc}
\qquad
  \ru{\CompTy \Gamma u {E_1} N \qquad
      \CompTy \Gamma t {E_2} N
    }{\CompTy \Gamma {\catch u e t} {\eee} N}
\]
If $u$ throws exception $e$, then ``$\catch u e t$'' computes $t$, else $u$.
%
In $\SET$ we use the graded monad $\T[E]\,A = E + A$ and a family of
morphism $\traise\,e : \SET(1,\T[\{e\}]A)$ to interpret
$\dent{\throw e} = \traise\,e \comp \langle\rangle$.  To interpret
$\tcatch$, define a family
$\thandle_N\,e : \Hom[\SET] {N_{E_1} \times N_{E_2}} {N_{\eee}}$ by
induction on $N \in \NTy$:
\[
\begin{array}{lcl}
  \handle[\diamond P] e g h & = & \left\{
                                  \begin{array}{ll}
                                    h & \mbox{if } g = \iota_1 e \\
                                    g & \mbox{otherwise}
                                  \end{array}
\right. \\
  \handle[P \To N] e g h \, a & = & \handle[N] e {(g\,a)} {(h\,a)} \\
  \handle[\tupt I N] e g h \, i & = & \handle[N_i] e {(g\,i)} {(h\,i)} \\
\end{array}
\]
Catching is then
$\dent {\catch u e t} = \handle[N] e \comp \langle \dent u, \dent t
\rangle$.
Note that $\thandle$ cannot be defined in terms of $\trun$, but we
have to break $N$ down to monadic type $\diamond P$
to get access to the exception
thrown by $u$.


\subsection{Digression: grading via a partial monoid}

Not all sequences of effects are always meaningful: For instance,
reading from a file before it was opened is impossible and could be
prevented statically by graded effect typing.  This could be modelled
by adding a maximal element $\top : \Eff$---with $e \leq \top$ for all
$e : \Eff$ that signifies an inconsistent state.  This error element
would also be dominant in sequences, \ie,
$\top \bu e = e \bu \top = \top$.
A program $\CompTy \Gamma t e N$ would only be accepted if $e \not= \top$.

Alternatively, we could work with a partial monoid, \ie, a carrier $\Eff$ with a predicate $\Ge{\leq}\_$ and a ternary relation $\_{\bu}\_{\leq}\_$ such that
the following laws hold:
\begin{enumerate}
\item Unit: $e \leq e' \defiff \exists e_0.\, \Ge \leq e_0 \land e_0 \bu e \leq e'$ iff $\exists e_0.\, \Ge \leq e_0 \land e \bu e_0 \leq e'$.
% \item Unit: $e \leq e' \defiff (\exists e_0.\, \Ge \leq e_0 \land e_0 \bu e \leq e')$ iff $(\exists e_0.\, \Ge \leq e_0 \land e \bu e_0 \leq e')$.
\item Associativity: $\exists e_{12}.\ e_1 \bu e_2 \leq e_{12} \land e_{12} \bu e_3 \leq e_{123}$ iff $\exists e_{23}.\ e_1 \bu e_{23} \leq e_{123} \land e_2 \bu e_3 \leq e_{23}$.
\item Monotonicity of $\Ge{\leq}\_$: If $\Ge \leq e$ and $e \leq e'$ then $\Ge \leq e'$.
\item Monotonicity of $\_{\bu}\_{\leq}\_$:  If $e_1' \leq e_1$ and $e_2' \leq e_2$ and $e \leq e'$ and $e_1 \bu e_2 \leq e$ then $e_1' \bu e_2' \leq e'$.
\item Reflexivity and transitivity of $\leq$.
\end{enumerate}
The typing rules for $\diamond$ would change accordingly.
\begin{gather*}
 \nrux{\rintro\diamond}
     {\ValTy \Gamma v P
    }{\CompTy \Gamma {\ret v} e {\diamond P}
    }{\graybox{\Ge \leq e}}
\qquad
 \nrux{\relim\diamond}
     {\CompTy \Gamma u {e_1} {\diamond P} \qquad
      \CompTy {\ext \Gamma x P} t {e_2} N
    }{\CompTy \Gamma {\bind x u t} e N
    }{\graybox{e_1 \bu e_2 \leq e}}
\end{gather*}
Rule $\rsub$ would be admissible.

A monad $\T : \Eff \to [\CC \to \CC]$ graded by a partial monoid
$\Eff$ has natural transformations $\Id \todot \T[e]$ for $\Ge \leq e$
and $\T[e_1] \circ \T[e_2] \todot \T[e]$ for $e_1 \bu e_2 \leq e$.  A
$\T$-algebra $B$ has morphism $\run[B] : \T[e_1]B_{e_2} \to B_{e}$
for $e_1 \bu e_2 \leq e$.

\begin{remark}
Another way of restricting effect composition is to use a 2-category
$\Eff$ where the objects $i,j,k$ are state types, the morphisms $e$
effect classifiers and the 2-cells effect subsumption $e \leq e'$.
The necessary theory has been worked out by
\citet{orchardWadlerEades:msfp20}.
%
An example would be a typed state monad
$\T[e: i \to j] A = S_i \to (A \times S_j)$ where the state type $S_i$
is indexed, and effects $e : i \to j$ may only make valid
modifications to the state.
%
For instance, the index could be a set of pointers denoting the
allocated heap cells and a read/write/deallocate effect would require
the respective pointer to be a member of this set (and remove it in
case of deallocate).
\end{remark}


\section{Coeffect-graded CBPV}
\label{sec:coeff}

\newcommand{\lolli}{\multimap}
\newcommand{\qfun}[3]{#1#2 \lolli #3}
\newcommand{\gqfun}[3]{\graybox{#1#2 \lolli #3}}
%\renewcommand{\CompTy}[3]{#1 \vdash #2 : #3}
\newcommand{\nbox}[1]{}
\newcommand{\tzip}{\mathsf{zip}}
\newcommand{\textract}{\mathsf{extract}}
\newcommand{\extract}[1][]{\textract_{#1}}
\newcommand{\tdisplay}{\mathsf{display}}
\newcommand{\display}[1][]{\tdisplay_{#1}}
% \newcommand{\tduplicate}{\mathsf{display}}
% \newcommand{\duplicate}[1][]{\tdisplay_{#1}}
\newcommand{\tduplicate}{\mathsf{duplicate}}
\newcommand{\duplicate}[1][]{\tduplicate_{#1}}
\newcommand{\tcobind}{\mathsf{cobind}}
\newcommand{\tbind}{\mathsf{bind}}
\newcommand{\tget}{\mathsf{get}}
\newcommand{\tmodify}{\mathsf{modify}}
\newcommand{\thead}{\mathsf{head}}
\newcommand{\ttail}{\mathsf{tail}}
\newcommand{\suc}{(\_{+}1)}
\newcommand{\Float}{\mathsf{Float}}
\newcommand{\tiso}{\mathsf{iso}}
\newcommand{\isot}{\tiso^\otimes}

CBPV places the monad $\diamond P$ at the transition from positive
types to negative types.  Dually, the transition $\Box N$ from
negative types to positive types is a vessel that can be filled with a
comonad.  Just like negative types $N$ are monad algebras, positive
types $P$ can be \emph{comonad coalgebras}.

\subsection{Comonadic CBPV and their comonad coalgebras}

% Let us consider the \emph{context comonad} $\D\,B = S \times B$ that
% gives us access to a value in $S$.
% \footnote{\url{https://bartoszmilewski.com/2017/01/02/comonads/}}
% Besides functoriality, a comonad
% has the natural transformations $\textract$ and $\display$, dual to
% $\treturn$ and $\tjoin$.  The implementation of the context comonad is
% as follows:
% \[
% \begin{array}{lcl}
%   \textract & : & \D\,B \to B \\
%   \textract\,(s,b) & = & b
% \\[1.5ex]
%   \tdisplay & : & \D,\B \to \D\,(\D\,B) \\
%   \tdisplay\,(s,b) & = & (s, (s, b))
% \\[1.5ex]
%   \tget & : & \D\,B \to S \\
%   \tget\,(s,b) & = & s
% \\[1.5ex]
%   \tmodify
% \end{array}
% \]
% The context comonad is not monoidal.

Let us consider the \emph{stream comonad}%
\footnote{\url{https://bartoszmilewski.com/2017/01/02/comonads/}}
$\D\,B = \bN \to B$,
with $\thead\,s = s\,0$ and $\ttail\,s = s \comp \suc$.
The stream comonad lets us work in a setting where values are not
single data points, but streams of data.
Besides functoriality, a
comonad has the natural transformations $\textract$ and $\display$
(in category theory called $\Ge$ and $\Gd$),
dual to $\treturn$ and $\tjoin$.  The implementation of the stream
comonad in $\SET$ is as follows:
\[
\begin{array}{lcl}
  \textract & : & \D\,B \to B \\
  \textract\,s & = & \thead\,s
\\[1.5ex]
  \tdisplay & : & \D\,B \to \D\,(\D\,B) \\
  \thead\,(\tdisplay\,s) & = & s \\
  \ttail\,(\tdisplay\,s) & = & \tdisplay\,(\ttail\,s) \\
\end{array}
\]
The generic operations of a comonad allow us to \emph{extract} the
value wrapped in a comonadic structure and to \emph{display} another
layer of that structure.

The comonad laws are a simple dualization of the monad laws:
\[
\xymatrix@C=10ex{
\D\,B  \ar@{=}[dr]
  & \D\,(\D\,B) \ar[l]_{\textract} \ar[r]^{\tdisplay}
  & \D\,(\D\,(\D\,B))
\\
  & \D\,B \ar[u]_{\tdisplay} \ar[r]^{\tdisplay} \ar@{=}[dr]
  & \D\,(\D\,B) \ar[u]_{\tfmap\,\tdisplay} \ar[d]^{\tfmap\,\textract}
\\
  &
  & \D\,B
}
\]

A \emph{monoidal} comonad implements a morphism $\tzip$ that combines
a tuples of comonadic values into one comonadic tuple:
\[
\begin{array}{lcl}
  \tzip & : & \tupt {i:I} {(\D\,B_i)} \to \D(\tupt I B) \\
  \tzip\,(s_i)_{i:I}\,n & = & (s_i\,n)_{i:I} \\
\end{array}
\]
Many other comonads are monoidal,
like the store comonad $(S \To \_) \times S$,
though not all, \eg, the context comonad $S{\times}\_$.
In this article, we consider only monoidal comonads.

If type constructor $\Box$ is interpreted as a monoidal comonad $\D$,
then all positive types $P$ can be interpreted as $\D$-coalgebras
$A = \den P$, \ie, implement a morphism $\expose[A] : A \to \D\,A$
satisfying the laws of a comonad coalgebra:
\[
\xymatrix@C=10ex{
A  \ar@{=}[dr]
  & \D\,A \ar[l]_{\textract} \ar[r]^{\tdisplay}
  & \D\,(\D\,A)
\\
  & A \ar[u]_{\texpose} \ar[r]^{\texpose}
  & \D\,A \ar[u]_{\tfmap\,\texpose}
}
\]
The implementation of $\expose[\den P]$ proceeds by induction on $P$.
\[
\begin{array}{lcl}
  \expose[\den{\Box N}] & : & {\D\,\den N} \to {\D\,(\D\,\den N)} \\
  \expose[\den{\Box N}] & = & \display
\\[1ex]
  \expose[\den{\tupt I P}] & : & \tupt {i:I} {\den{P_i}} \to \D\,(\tupt {i:I} {\den{P_i}}) \\
  \expose[\den{\tupt I P}] & = & \tzip \comp \tupt {i:I} \expose[\den{P_i}]
\\[1ex]
  \expose[\den{\sumt I P}] & : & \sumt {i:I} {\den{P_i}} \to \D\,(\sumt {i:I} {\den{P_i}}) \\
  \expose[\den{\sumt I P}] & = & \left[\tfmap\,\iota_i \comp \expose[\den{P_i}]\right]_{i:I} \\
\end{array}
\]
Positive base types $o \in \PTy$ are required to be $\D$-coalgebras.
For instance, elements of a base type like
$\Float \in \PTy$ could represent streams of floating point numbers,
\eg, continuous measurements from a sensor.

Because comonad coalgebras are closed under products for monoidal
coalgebras, contexts $\Gamma$ are interpreted as comonad coalgebras
$\den\Gamma$.  In the presence of a comonad interpretation of $\Box$,
the operations $\tthunk$ and $\tforce$ are no longer the identity, but
``generalized cobind'' and $\extract$:
\[
\begin{array}{lllcl}
 \rintro\Box
  & \ru{\oCompTy \Gamma t N
      }{\ValTy \Gamma {\thunk t} {\Box N}}
  & \dent{\thunk t} & = & \fmap[\D]{\dent t} \comp \expose[\den\Gamma]
\\[3ex]
 \relim\Box
  & \ru{\ValTy \Gamma v {\Box N}
      }{\oCompTy \Gamma {\force v} N}
  & \dent{\force v} & = & \textract \comp \dent v
\end{array}
\]
The map $\dent\tthunk : \Hom A B \to \Hom A {\D\,B}$ generalizes
$\tcobind : \Hom{\D\,C} B \to \Hom{\D\,C}{\D\,B}$ to $\D$-coalgebras
$A$ in the same way that the monadic
$\tbind : \Hom{A}{\T\,D} \to \Hom{\T\,A}{\T\,D}$ is generalized to
$\Hom{A}{B} \to \Hom{\T\,A}{B}$ for $\T$-algebras $B$ in CBPV.


% In a stream-valued instance of CBPV, elements of a base type like
% $\Float \in \PTy$ would represent streams of floating point numbers,
% \eg, continuous measurements from a sensor.
% However, a single comonad $\D$ may not sufficient to express what we
% want.  In the next section, we look at \emph{graded comonads}.

\subsection{Graded comonads and their coalgebras}

\emph{Graded comonads} have been utilized to give semantics to
context-dependent computation \citep{orchard:icfp14}.  For grading,
\loccit uses a \emph{resource algebra} in form of a preordered
semiring
$(\R,{+},0,{\cdot},1,{\leq})$.
% $(\R,\_{+}\_,0,\_{\cdot}\_,1,\leq)$.
%
The semantics is \emph{resource aware} and thus not constructed in a
cartesian-closed category but in a symmetric monoidal closed category
$(\CC,\otimes,\I,\lolli)$.
Introduction $\Lambda$ and elimination $\teval$ of
exponentials are there
$\Lambda : \Hom {C \otimes A} B \to \Hom C {A \lolli B}$
and $\teval : \Hom{(A \lolli B) \otimes A} B$.  The tensor product
$\otimes$ is a bifunctor and it is customary to overload $\otimes$ for
the functorial action
$f_1 \otimes f_2 : \Hom{A_1 \otimes A_2}{B_1 \otimes B_2}$ where
$f_i : \Hom{A_i}{B_i}$.  We shall also use $I$-ary products
$\tupt {i:I} {A_i}$ and their functorial action $\tupt {i:I} {f_i}$.
Further, the symmetric monoidal structure is usually witnessed by
natural
isomorphisms for left $\lambda_A : \Hom {\I \otimes A} A$ and right
unit $\rho_A : \Hom {A \otimes I} A$, associativity
$\Ga_{A,B,C} : \Hom{A \otimes (B \otimes C)} {(A \otimes B) \otimes
  C}$
and swap $\sigma_{A,B} : \Hom {A \otimes B} {B \otimes A}$.  We shall
summarize combinations of these isomorphisms under the name
$\isot : \Hom {\tupt I A} {\tupt J B}$ when the multisets
$\{A_i\}_{i:I}$ and $\{B_j\}_{j:J}$ coincide modulo addition and
deletion of units ($\I$).


As a running example for resource accounting,
we use the semiring $\R = \Pot\bN \setminus \emptyset$ of
\emph{multiplicities} ordered by $\supseteq$ with pointwise sum and
product, \eg, $r + s = \{ n + m \mid n \in r, m \in s \}$.
Subsemirings of $\R$ have been used for quantitative typing
\cite{sergeyVytiniotisPeytonJones:popl14,mcBride:wadler60,atkey:lics18}.
The order expresses precision of the quantities, \eg,
$\{1\} \geq \{0,1\} \geq \bN$ states that linear use of a variable is more informative than affine use than unrestricted use.

A matching symmetric monoidal closed category $\CC$ can be obtained as
follows:
% Assume a preordered commutative monoid $(\W,\oplus,\emp,\wleq,\winf)$
% that is also a $\winf$-semilattice with distribution law $\winf_{i:I} (w_i \oplus w'_i) \wleq \winf_I w \oplus \winf_I w'$.  The elements (``worlds'') can be
% seen as collection of available resources under uncertainty.  The
% order $w \wleq w'$ again expresses precision of information, \ie, if
% we can construct something from resources $w$ we can also construct it
% from $w'$.  The infimum $\winf_{i:I}w_i$ expresses we can build the
% ``thing'' from any of the $w_i$.  Multiplication $n \cdot w$ is
% understood as $w \oplus \dots \oplus w$ with $n$ summands.
% %
% Given a commutative monoid $(\M,\uplus,\emptyset)$ whose elements can
% be thought of as bags of atomic resources, an instance for $\W$ would
% be $\W = \Pot \M$ under ${\wleq} = {\subseteq}$ and pointwise union
% $w \oplus w' = \{ m \uplus m' \mid m \in w, m' \in w' \}$.
%
% Resource qualifiers $r \in \R$ operate on worlds via $r \cdot w = \winf_{n \in r} (n \cdot w)$.  It is routine to verify that $\W$ is almost a left semimodule to $\R$, \ie, the following laws hold:
% \[
% \begin{array}{rclclcl}
%   1 \cdot w
%     & = & \winf_{n \in \{1\}} nw
%     & = & 1w
%     & = & w
% \\
%   (rs) \cdot w
%     & = & \winf_{n \in r, m \in s} (nm)w
%     & = & \winf_{n \in r} n(\winf_{m \in s} mw)
%     & = & r \cdot (s \cdot w)
% \\
%   0 \cdot w
%     & = & \winf_{n \in \{0\}} nw
%     & = & 0w
%     & = & \emp
% \\
%   (r + s) \cdot w
%     & = & \winf_{n \in r, m \in s} (n+m)w
%     & = & \winf_{n \in r} \winf_{m \in s} (nw \oplus mw)
%     & = & (r \cdot w) \oplus (s \cdot w)
% \\
%   r \cdot \emp
%     & = & \winf_{n \in r} n\emp
%     & = & \winf_{n \in r} \emp
%     & = & \emp
% \\
%   r \cdot (w \oplus w')
%     & = & \winf_{n \in r} n(w \oplus w')
%     & \wleq & \winf_{n \in r} nw \oplus \winf_{n \in r} nw'
%     & = & (r \cdot w) \oplus (r \cdot w')
% \\
% \end{array}
% \]
Assume a preordered commutative monoid $(\W,\oplus,\emp,\wleq,\wsup)$
that is also a $\wsup$-semilattice with distribution law $\wsup_{i:I} (w_i \oplus w'_i) \wleq \wsup_I w \oplus \wsup_I w'$.  The elements (``worlds'') can be
seen as collection of available resources under choice.  The
order $w \wleq w'$ again expresses precision of information, \ie, if
we can construct something from resources $w$ we can also construct it
from $w'$ where we have additional choices.
The supremum $\wsup_{i:I}w_i$ expresses we can build the
``thing'' from any of the $w_i$.  Multiplication $n \cdot w$ is
understood as $w \oplus \dots \oplus w$ with $n$ summands.
%
Given a commutative monoid $(\M,\uplus,\emptyset)$ whose elements can
be thought of as bags of atomic resources, an instance for $\W$ would
be $\W = \Pot \M$ under ${\wleq} = {\subseteq}$ and pointwise union
$w \oplus w' = \{ m \uplus m' \mid m \in w, m' \in w' \}$.

Resource qualifiers $r \in \R$ operate on worlds via
$r \cdot w = \wsup_{n \in r} (n \cdot w)$.
%
For instance the \emph{affine} qualifier $\{0,1\}w = \emp \oplus w$
gives us the choice of using resources $w$ or not ($\emp$).
%
It is routine to verify that $\W$ is almost a left semimodule to $\R$,
\ie, the following laws hold:
\[
\begin{array}{rclclcl}
  1 \cdot w
    & = & \wsup_{n \in \{1\}} nw
    & = & 1w
    & = & w
\\
  (rs) \cdot w
    & = & \wsup_{n \in r, m \in s} (nm)w
    & = & \wsup_{n \in r} n(\wsup_{m \in s} mw)
    & = & r \cdot (s \cdot w)
\\
  0 \cdot w
    & = & \wsup_{n \in \{0\}} nw
    & = & 0w
    & = & \emp
\\
  (r + s) \cdot w
    & = & \wsup_{n \in r, m \in s} (n+m)w
    & = & \wsup_{n \in r} \wsup_{m \in s} (nw \oplus mw)
    & = & (r \cdot w) \oplus (s \cdot w)
\\
  r \cdot \emp
    & = & \wsup_{n \in r} n\emp
    & = & \wsup_{n \in r} \emp
    & = & \emp
\\
  r \cdot (w \oplus w')
    & = & \wsup_{n \in r} n(w \oplus w')
    & \wleq & \wsup_{n \in r} nw \oplus \wsup_{n \in r} nw'
    & = & (r \cdot w) \oplus (r \cdot w')
\\
\end{array}
\]
Because the last law is not an equality but just the inequality
$r (w \oplus w') \wleq rw \oplus rw'$, we have ``almost'' a
semimodule.

Objects of $\CC$ are functors $A : [(\W,\wleq) \to \SET]$ and morphisms
$f : \Hom A B$ are natural transformations
$(f_w : A_w \to B_w)_{w:W}$, \ie,
$B_{w \wleq w'} \comp f_w = f_{w'} \comp A_{w \wleq w'}$.  The tensor
$A \otimes B$ is Day's convolution
$(A \otimes B)_w = \bigcup_{w_1 \oplus w_2 \wleq w} (A_{w_1} \times B_{w_2})$
% $(A \otimes B)_w = \bigcup_{w_1,w_2 \mid w \geq w_1 \oplus w_2} A_{w_1} \times A_{w_2}$
with unit $\I_w = \bigcup_{\emp \wleq w} 1$.  The unit $\I$ is
constructible at world $w$ from nothing (the unit set $1$) if $\emp \wleq w$,
\ie, if the world $w$ includes the choice of using no resources.  A
tensor $A \otimes B$ is constructible at world $w$ if $w$ includes the
choice to split the resources into $w_1$ and $w_2$ to construct $A$
and $B$, resp.  The exponential $A \lolli B$ is determined by currying
$C \otimes A \lolli B \cong C \lolli (A \lolli B)$ and given by
$(A \lolli B)_w = \bigcap_{w_1 \oplus w \wleq w_2} (A_{w_1} \to
B_{w_2})$.

The thus constructed symmetric monoidal closed category $\CC$ has a
$\R$-graded comonad
\[
\begin{array}{lcl}
  \D & : & [(\R,\leq) \to [\CC \to \CC]] \\
(\D[r] A)_w & = &
  \bigcup_{rw' \wleq w} A_{w'}
  % \bigcup_{w' \mid w \wleq rw'} A_{w'}
\end{array}
\]
implementing trivially the following natural transformations.
Herein, we use as second monoidal structure
on the functor category $[\CC \to CC]$ the
pointwise tensor product of functors
$(\dot\otimes_{i:I}F_i)\,A = \otimes_{i:I}(F_i\,A)$ in
$[\CC \to \CC]$, in particular $\dot\I\,A = \I$.
\[
\begin{array}{lcl@{~}c@{~}l}
  \textract   & : & \D[1]   & \todot & \Id \\
  \tdisplay & : & \D[rs]  & \todot & \D[r] \comp \D[s] \\
  \tdrop    & : & \D[0]   & \todot & \dot{\I} \\
  \tduplicate  & : & \D[r+s] & \todot & \D[r] \mathbin{\dot\otimes} \D[s] \\
  % \tdrop    & : & \Hom {\D[0]\,A} A \\
  % \tduplicate  & : & \Hom {\D[r+s]\,A} {\D[r]\,\A \otimes \D[s]\,A} \\
\end{array}
\]
For example, $\tduplicate$ takes
$a \in \bigcup_{rw' \oplus sw' \wleq w} A_{w'}$ to
$(a,a) \in \bigcup_{w_1 \oplus w_2 \wleq w} \bigcup_{rw_1' \leq w_1}
\bigcup_{sw_2' \leq w_2} A_{w'_1} \times A_{w'_2}$.
The intermediate worlds are $w_1' = w_2' = w'$ and $w_1 = rw'$ and $w_2 = sw'$.

The monoidal character of $\D$ is witnessed by the trivial morphism $\tzip$:
\[
%\newcommand{\Aiw}{(A_i)_{w'_i}}
\newcommand{\Aiw}{A_{iw'_i}}
\begin{array}{lcl}
  \tzip       & : & \Hom {\tupt {i:I} {(\D[r]\,A_i)}} {\D[r]\,(\tupt I A)} \\
  \tzip_w     & : & \bigcup_{\oplus_{i:I}w_i \wleq w} \Pi_{i:I} (\D[r]\,A_i)_{w_i} \to \bigcup_{rw' \wleq w} (\tupt I A)_{w'} \\
  \tzip_w     & : & \bigcup_{\oplus_{i:I} w_i \wleq w} \Pi_{i:I} (\bigcup_{rw'_i \wleq w_i} \Aiw) \to \bigcup_{rw' \wleq w} \bigcup_{\oplus_{i:I} w'_i \wleq w'} \Pi_{i:I} \Aiw \\
   \tzip_w\,(a_i)_{i:I} & = & (a_i)_{i:I} \\
\end{array}
\]

The comonad laws generalize to graded comonads as follows:
\[
\xymatrix@C=10ex{
\D[r]\,B  \ar[d]_{\tdisplay} \ar[r]^{\tdisplay} \ar@{=}[dr]
  & \D[r]\,(\D[1]\,B)  \ar[d]^{\tfmap\,\textract}
\\
\D[1]\,(\D[r]\,B) \ar[r]^{\textract}
  & \D[r]\,B
}
\qquad
\xymatrix@C=12ex{
\D[qrs]\,B \ar[d]_{\tdisplay} \ar[r]^{\tdisplay}
  & \D[qr]\,(\D[s]\,B) \ar[d]^{\tdisplay}
\\
\D[q]\,(\D[rs]\,B) \ar[r]^{\tfmap\,\tdisplay}
  & \D[q]\,(\D[r]\,(\D[s]\,B))
}
\]
% \[
% \xymatrix@C=10ex{
% \D[r]\,B  \ar@{=}[dr]
%   & \D[1]\,(\D[r]\,B) \ar[l]_{\textract}
% \\
%   & \D[r]\,B \ar[u]_{\tdisplay}
% }
% \qquad
% \xymatrix@C=10ex{
% \D[r]\,B  \ar[r]^{\tdisplay} \ar@{=}[dr]
%   & \D[r]\,(\D[1]\,B)  \ar[d]^{\tfmap\,\textract}
% \\
%   & \D[r]\,B
% }
% \]
% \[
% \xymatrix@C=10ex{
% \D[q]\,(\D[rs]\,B) \ar[r]^{\tdisplay}
%   & \D[q]\,(\D[r]\,(\D[s]\,B))
% \\
% \D[qrs]\,B \ar[u]_{\tdisplay} \ar[r]^{\tdisplay}
%   & \D[qr]\,(\D[s]\,B) \ar[u]_{\tfmap\,\tdisplay}
% }
% \]
These laws reflect that $\D$ maps the multiplicative monoid
$(\R,{\cdot},1)$ to the monoid structure $([\CC \to \CC],{\circ},\Id)$
of composition in the functor category $[\CC \to \CC]$.  Similar laws
need to hold for the additive monoid and distributivity.
\[
\xymatrix@C=12ex{
\D[0]\,B \otimes \D[r]\,B \ar[d]_{\tdrop \otimes \tid}
& \D[r]\,B \ar[l]_{\tduplicate} \ar[r]^{\tduplicate} \ar@{=}[d]
& \D[r]\,B \otimes \D[0]\,B \ar[d]^{\tid \otimes \tdrop}
\\
\I \otimes \D[r]\,B \ar[r]^{\Gl}
& \D[r]\,B
& \D[r]\,B \otimes \I \ar[l]_{\Gr}
}
\]
\[
\xymatrix@C=10ex{
\D[q]\,B \otimes \D[r+s]\,B  \ar[d]_{\tid \otimes \tduplicate}
& \D[q+r+s]\,B \ar[l]_{\tduplicate} \ar[r]^{\tduplicate}
& \D[q+r]\,B \otimes \D[s]\,B \ar[d]^{\tduplicate \otimes \tid}
\\
\D[q]\,B \otimes (\D[r]\,B \otimes \D[s]\,B) \ar[rr]^{\Ga}
& & (\D[q]\,B \otimes \D[r]\,B) \otimes \D[s]\,B
}
\]
\[
\xymatrix@C=22ex{
\D[(q+r)s]\,B \ar[r]^{\tdisplay} \ar[d]_{\tduplicate}
& \D[q+r]\,(\D[s]\,B) \ar[d]^{\tduplicate}
\\
\D[qs]\,B \otimes \D[rs]\,B \ar[r]^{\tdisplay \otimes \tdisplay}
& \D[q](\D[s]\,B) \otimes \D[r](\D[s]\,B)
}
\]
\[
\xymatrix{
\D[qr]\,B \otimes \D[qs]\,B \ar[d]_{\display \otimes \display}
  & \D[q(r+s)]\,B \ar[l]_{\tduplicate} \ar[r]^{\display}
  & \D[q](\D[r+s]\,B) \ar[d]^{\fmap\tduplicate}
\\
\D[q](\D[r]\,B) \otimes \D[q](\D[s]\,B) \ar[rr]^{\tzip}
&&
\D[q](\D[r]\,B \otimes \D[s]\,B)
}
\]

A $\D$-coalgebra is a functor $A : [(\R,\leq) \to \CC]$ with family of
morphisms $\expose[r,s] : A_{rs} \to \D[r]\,A_s$ natural in $r$ and
$s$.  For the $\D$-coalgebra $\D[\_]\,B$ the family $\expose$ is just
$\tdisplay$, and the suitably generalizable laws for $\tdisplay$ are
required to hold for $\texpose$:
\[
\xymatrix@C=10ex{
A_r  \ar[d]_{\texpose} \ar@{=}[dr]
\\
\D[1]\,A_r \ar[r]^{\textract}
  & A_r
}
\qquad
\xymatrix@C=12ex{
A_{qrs} \ar[d]_{\texpose} \ar[r]^{\texpose}
  & \D[qr]\,A_s \ar[d]^{\tdisplay}
\\
\D[q]\,A_{rs} \ar[r]^{\tfmap\,\texpose}
  & \D[q]\,(\D[r]\,A_s)
}
\]
Further, $A$ needs to map the additive monoidal structure $(\R,+,0)$
to the monoidal structure $(\CC,\otimes,\I)$ associated with the
tensor product in $\CC$.  We choose to overload the names
$\tduplicate$ and $\tdrop$ here to accommodate for the generalization
from $\R$-comonads to $\R$-comonad algebras:
\[
\begin{array}{lcl}
  \tduplicate & : & \Hom {A_{r+s}} {A_r \otimes A_s} \\
  \tdrop      & : & \Hom {A_0} {\I} \\
\end{array}
\]
The laws of $\tduplicate$ and $\tdrop$ for comonadic objects
$\D[r]\,B$ immediately generalize to comonad algebras $A_r$:
\[
\xymatrix@C=12ex{
A_{0} \otimes A_r \ar[d]_{\tdrop \otimes \tid}
& A_r \ar[l]_{\tduplicate} \ar[r]^{\tduplicate} \ar@{=}[d]
& A_r \otimes A_{0} \ar[d]^{\tid \otimes \tdrop}
\\
\I \otimes A_r \ar[r]^{\Gl}
& A_r
& A_r \otimes \I \ar[l]_{\Gr}
}
\]
\[
\xymatrix@C=10ex{
A_q \otimes A_{r+s}  \ar[d]_{\tid \otimes \tduplicate}
& A_{q+r+s} \ar[l]_{\tduplicate} \ar[r]^{\tduplicate}
& A_{q+r} \otimes A_s \ar[d]^{\tduplicate \otimes \tid}
\\
A_q \otimes (A_r \otimes A_s) \ar[rr]^{\Ga}
& & (A_q \otimes A_r) \otimes A_s
}
\]
\[
\xymatrix@C=22ex{
A_{(q+r)s} \ar[r]^{\texpose} \ar[d]_{\tduplicate}
& \D[q+r]\,A_s \ar[d]^{\tduplicate}
\\
A_{qs} \otimes A_{rs} \ar[r]^{\texpose \otimes \texpose}
& \D[q]A_s \otimes \D[r]A_s
}
\]
\[
\xymatrix{
A_{qr} \otimes A_{qs} \ar[d]_{\expose \otimes \expose}
  & A_{q(r+s)} \ar[l]_{\tduplicate} \ar[r]^{\expose}
  & \D[q]\,A_{r+s} \ar[d]^{\fmap\tduplicate}
\\
\D[q]A_r \otimes \D[q]A_s \ar[rr]^{\tzip}
&&
\D[q](A_r \otimes A_s)
}
\]

Graded comonad algebras are closed under pointwise sums $(\sumt I A)_r = \sumt {i:I} {A_{i,r}}$ and products $(\tupt I A)_r = \tupt {i:I} {A_{i,r}}$.
\[
\begin{array}{lcl}
  \expose[\sumt I A] & : & \Hom {(\sumt I A)_{rs}} {\D[r](\sumt I A)_s} \\
  \expose[\sumt I A] & = & \left[  \fmap{\iota_i} \comp \expose[A_i] \right]_{i:I}
\\[1ex]
  \tdrop[\sumt I A] & : & \Hom {(\sumt I A)_0} \I \\
  \tdrop[\sumt I A] & = & \left[ \tdrop[A_i] \right]_{i:I}
\\[1ex]
  \duplicate[\sumt I A] & : & \Hom {(\sumt I A)_{r+s}} {(\sumt I A)_r \otimes (\sumt I A)_s} \\
  \duplicate[\sumt I A] & = & \left[ (\iota_i \otimes \iota_i) \comp \duplicate[A_i] \right]_{i:I}
\\[2ex]
  \expose[\tupt I A] & : & \Hom {(\tupt I A)_{rs}} {\D[r](\tupt I A)_s} \\
  \expose[\tupt I A] & = & \tzip \comp \tupt {i:I} {\expose[A_i]}
\\[1ex]
  \tdrop[\tupt I A] & : & \Hom {(\tupt I A)_0} \I \\
  \tdrop[\tupt I A] & = & \isot \comp \tupt {i:I} {\tdrop[A_i]}
\\[1ex]
  \duplicate[\tupt I A] & : & \Hom {(\tupt I A)_{r+s}} {(\tupt I A)_r \otimes (\tupt I A)_s} \\
  \duplicate[\tupt I A] & = & \isot \comp \tupt {i:I} {\duplicate[A_i]}
\end{array}
\]
This enables us to interpret all positive types of CBPV as graded
comonad coalgebras.


\subsection{Structured graded comonads and their algebras}

Coeffect, quantitative, and many modal type systems maintain a typing
context where each variable $x$ is annotated by a resource qualifier
$r$ in addition to its type $P$.  Such a typing context
$(r_1x_1{:}P_1, \dots, r_nx_n{:}P_n) =: \gamma\Gamma$ can be
split into a pure typing context
$\Gamma = x_1{:}P_1, \dots, x_n{:}P_n$ and a resource context
$\gamma = x_1{:}r_1, \dots, x_n{:}r_n$ such that
$\dom \gamma = \dom \Gamma$.  We shall freely mix the two notations as
it suits our purpose.

In coeffect-graded CBPV, we wish to interpret each type in the context
as a $\D$-coalgebra for a fixed graded comonad $\D$.  The
interpretation $\den \Gamma$ of the context $\Gamma$ should then be a
comonad coalgebra over the grading $\gamma$ such that judgements
$\qCompTy \gamma \Gamma t N$ can be interpreted as morphisms
$\dent t : \Hom {\Den\Gamma \gamma} {\denn N}$.  Since each variable
comes with its own resource qualifier, we cannot simply model the
context as a tensor product $\tupt {x:\dom\Gamma} \den{\Gamma(x)}$ of
$\D$-coalgebras since this product would be indexed by a single resource
qualifier $r : \R$ rather than a resource context
$\gamma : \dom\Gamma \to \R$.  The solution offered by
\citet{orchard:icfp14} are \emph{structured indexed comonads}.  We
shall generalize this to comonad coalgebras to the extend needed for
interpreting contexts.

\citet{mcBride:wadler60} observed that resource contexts
$\gamma : \R^I$ form a left $\R$-semimodule under pointwise addition
$(\gamma + \delta)(i) = \gamma(i) + \delta(i)$ and scaling
$(r \cdot \gamma)(i) = r \cdot \gamma(i)$.  Contexts $\Gamma$ can thus
be interpreted as
(\emph{structured} or)
\emph{$\R^{\dom \Gamma}$-graded}
$\D$-coalgebras $C$ with
the following operations:
\[
\begin{array}{lcl}
  \tdrop[C]     & : & \Hom {C_0} {\I} \\
  \duplicate[C] & : & \Hom {C_{\gamma+\delta}} {C_\gamma \otimes C_\delta} \\
  \duplicate[C] & : & \Hom {C_{\sumt I \gamma}} {\tupt {i:I} C_{\gamma_i}} \\
  \expose[C]    & : & \Hom {C_{r\gamma}} {\D[r]\,C_\gamma} \\
\end{array}
\]
The second, generalized form of $\tduplicate$ will be used to split
resources accumulated from $I$ parties.  Note that
$A_r := C_{r\gamma}$ flattens a structured $\D$-coalgebra $C$ into an
ordinary one, $A$.

We will also need to interpret context extension.  To this end, we
shall employ a structured product
$(C \boxtimes D)_{\gamma.\delta} = C_\gamma \otimes D_\delta$ where
$\gamma : \R^I$ and $\delta : \R^J$ and thus $C \boxtimes D$ is a
$\R^{I+J}$-graded $\D$-coalgebra.  Further, we implicitly use the
isomorphism between $1$-structured $\D$-coalgebras $C : \R^1 \to \CC$
and $\R$-graded $\D$-coalgebras $A : \R \to \CC$ and define
\[
\den{\ext \Gamma x P}
  = \den \Gamma \boxtimes \denp P
  ,
\]
sweeping name issues under the carpet (to be handled by de Bruijn indices).


\begin{figure}[htbp]
\flushleft
\ruler{}
Types.
\[
\begin{array}{lllrl@{\qquad}l}
\PTy & \ni & P
  & ::= & \Box N
     \mid o \mid \sumty i I P \mid \tupty i I P
  & \nbox{Value types (positive types)} \\
\NTy & \ni & N
  & ::= & %\diamond P %
          \graybox{\compty r P}
     \mid \gqfun r P N \mid \recty i I N
  & \nbox{Computation types (negative types)} \\
\Cxt & \ni & \Gamma
  & ::= & \cempty \mid \ext \Gamma x P % \mid \graybox{\ext \Gamma y N}
  & \nbox{Typing context}
\\
\R & \ni & \graybox{r,s}
   & ::= & 0 \mid 1 \mid r + s \mid rs
   & \mbox{Resource qualifiers}
\\
\RCxt & \ni & \graybox{\gamma,\delta}
  & ::= & \cempty \mid \ext \gamma x r % \mid \ext \gamma y r
  & \mbox{Resource context}
\end{array}
\]
\dashruler{}
Terms.
\[
\begin{array}{lllrl@{\qquad}l}
\PTm & \ni & v,w
  & ::= & x
     \mid \thunk t
     \mid \inj i v
     \mid \ptup{\bar v}
  & \nbox{Values (positive terms)} \\
\NTm & \ni & t,u
  & ::= &
     %     \graybox{y} \mid \graybox{\qlet r y u t}
     % \mid
         \qlet {\graybox r} x v t
  & \nbox{Computations (negative terms):}
\\ &&& \mid &
         \force v
         % \graybox{\qforced r v y t}
    \mid \qcase {\graybox r} v {\overline{x.\,t}}
    \mid \qsplits {\graybox r} v {\bar x} t
  & \nbox{value eliminations}
\\ &&& \mid &
         \ret v
    \mid \bind x u t
    % \mid \qbind {\graybox r} x u t
  & \nbox{monad operations}
\\ &&& \mid &
         \lam x t      \mid \app t v
  & \nbox{functions}
\\ &&& \mid &
         \trecord\{\overline{i:t}\} % \ntup{\bar t}
       \mid \proj i t
  & \nbox{lazy tuples (records)}
\end{array}
\]
\dashruler{}
Value typing \fbox{$\gqValTy \gamma \Gamma v P$}\,.
\begin{gather*}
 \nru{\rvar %^+
    }{
    }{\ValTy {\gqext 0\Gamma 1x P} x P}
\qquad
 \nru{\rintro\Box}
     {\qCompTy \gamma \Gamma t N
    }{\qValTy \gamma \Gamma {\thunk t} {\Box N}}
\\[1.5ex]
 \nru{\rintro\GS}
     {\qValTy \gamma \Gamma v {P_i}
    }{\qValTy \gamma \Gamma {\inj i v} {\sumt I P}}
\qquad
 \nru{\rintro\otimes}
     {\forall i{:}I, \ \gqValTy {\gamma_i} \Gamma {v_i} {P_i}
    }{\gpValTy {\GS_{i:I}\gamma_i} \Gamma {\ptup v} {\tupt I P}}
\end{gather*}
\dashruler{}
Computation typing \fbox{$\gqCompTy \gamma \Gamma t N$}\,.
%\vspace{-2ex}
\begin{gather*}
% \graybox{
%  \nru{\rvar^-}{}
%      {\ValTy {\gqext 0 \Gamma 1 y N} y N}
% }
% \qquad
% \graybox{
%  \nru{\rlet^-}
%      {\qCompTy \delta \Gamma u N \qquad
%       \oCompTy {\qext \gamma \Gamma r y N} t {N'}
%     }{\gpCompTy {\gamma + r\delta} \Gamma {\qlet r y u t} {N'}}
% }
% \\[1.5ex]
%  \nru{\rlet^+}
 \nru{\rlet}
     {\gqValTy \delta \Gamma v P \qquad
      \oCompTy {\gqext \gamma \Gamma r x P} t N
    }{\gpCompTy {\gamma + r\delta} \Gamma {\gqlet r x v t} N}
\qquad
 \nru{\relim\Box}
     {\qValTy \gamma \Gamma v {\Box N}
    }{\qCompTy \gamma \Gamma {\force v} {N}}
 % \nru{\relim\Box}
 %     {\gqValTy \delta \Gamma v {\Box N} \qquad
 %      \oCompTy {\gqext \gamma \Gamma r y N} t {N'}
 %    }{\gpCompTy {\gamma + r\delta} \Gamma {\gqforced r v y t} {N'}}
\\[1.5ex]
 \nru{\relim\GS}
     {\gqValTy \delta \Gamma v {\sumt I P} \quad
      \forall i{:}I,\ \oCompTy {\gqext \gamma \Gamma r {x_i} {P_i}} {t_i} N
    }{\gpCompTy {\gamma + r\delta} \Gamma {\caser v {\{\graybox{r} x_i.\, t_i\}_{i:I}}} N}
\quad
 \nru{\relim\otimes}
     {\gqValTy \delta \Gamma v {\tupt I P} \quad
      \gqCompTy \gamma {\extr \Gamma {\overline{\graybox{r}x_i{:}P_i}^{i:I}}} t N
    }{\gpCompTy {\gamma + r\delta} \Gamma {\gqsplits r v {\bar x} t} N}
\\[1.5ex]
 \nru{\rintro\diamond}
     {\qValTy \gamma \Gamma v P
    }{\qCompTy {\graybox r\gamma} \Gamma {\ret v} {\compty {\graybox r} P}}
\qquad
 \nru{\relim\diamond}
     {\gqCompTy \delta \Gamma u {\compty {\graybox r} P} \qquad
      \oCompTy {\gqext \gamma \Gamma r x P} t  N
    }{\gpCompTy {\gamma + \delta} \Gamma {\bind x u t} N}
%  \nru{\rintro\diamond}
%      {\qValTy \gamma \Gamma v P
%     }{\qCompTy \gamma \Gamma {\ret v} {\diamond P}}
% \qquad
%  \nru{\relim\diamond}
%      {\gqCompTy \gamma \Gamma u {\diamond P} \qquad
%       \CompTy {\gqext \delta \Gamma r x P} t  N
%     }{\gpCompTy {r\gamma + \delta} \Gamma {\gqbind r x u t} N}
\\[1.5ex]
 \nru{\rintro\lolli}
     {\oCompTy {\gqext \gamma \Gamma r x P} t N
    }{\gqCompTy \gamma \Gamma {\lam x t} {\gqfun r P N}}
\qquad
 \nru{\relim\lolli}
     {\gqCompTy \gamma \Gamma t {\gqfun r P N} \qquad
      \gqValTy \delta \Gamma v P
    }{\gpCompTy {\gamma + r\delta} \Gamma {\app t v} N}
\\[1.5ex]
 \nru{\rintro\Pi}
     {\forall i{:}I,\ \qCompTy \gamma \Gamma {t_i} {N_i}
    }{\qCompTy \gamma \Gamma {\trecord {\{i : t_i\}_{i:I}}} {\rect I N}}
    % }{\qCompTy \gamma \Gamma {\ntup{\overline{t_i}^{i:I}}} {\rect I N}}
\qquad
 \nru{\relim\Pi}
     {\qCompTy \gamma \Gamma t {\rect I N}
    }{\qCompTy \gamma \Gamma {\proj i t}  {N_i}}
%\\[1.5ex]
\qquad
\graybox{
 \nrux{\rweak}
      {\gqCompTy {\gamma'} \Gamma t N}
      {\gqCompTy \gamma \Gamma t N}
      {\gamma \leq \gamma'}
}
\end{gather*}
\rule{\textwidth}{0.2pt}
  \caption{Coeffect-graded call-by-push-value.}
  \label{fig:coeff-cbpv}
\end{figure}


\subsection{Coeffect-graded CBPV: syntax and typing}

We now have the mathematical structures in place to define a
coeffect-graded variant of CBPV (see Fig.~\ref{fig:coeff-cbpv}).
The difference to pure CBPV is laid out in \emph{gray boxes}.

The fundamental novelty is that the typing judgements
$\qValTy \gamma \Gamma v P$ and $\qCompTy \gamma \Gamma t N$ are
equipped with resource contexts $\gamma$ matching the typing contexts
$\Gamma$.  A common pattern in the rules is that resource requirements
of the subterms are added when both subterms are or maybe evaluated at
runtime (rules \rintro\otimes, \rlet, %$\rlet^{+/-}$,
\relim{%\Box/
\GS/\otimes/\diamond/\lolli}).  Note that in \relim\GS,
the branches $t_i$ of the case statement (\relim\GS) share a resource
context $\gamma$ since only one of the branches is executed at
runtime.  Via rule $\rweak$,
the different resource requirements of the branches can be
subsumed under their maximum wrt.\ $\leq_\R$.  Similarly, the
components $t_i$ of a record (\rintro\Pi) share a resource context
since projection only retrieves one of the components (\relim\Pi).  This is
the opposite of eager tuples (\rintro\otimes) where the elimination
makes all components available at the same time (\relim\otimes), thus,
all components have to be evaluated at runtime.

Function types $\qfun r P N$ are now graded by a resource qualifier
$r$ that specifies how the function argument is to be used in the
function body.  (This type is often written $!_{r}P \lolli N$.)  In
quantitative typing with $\R = \Pot \bN$, qualifier $r$ gives the
possible usage quantities of the function argument, e.g. $\{1\}$ for
exactly one use (linear), $\{0,1\}$ for at most one use (affine),
$\{0\}$ for no use (constant), and $\bN$ for arbitrary use
(unrestricted).  A resource qualifier could also be a security level
(public or private), or a sensitivity level (a non-negative real)
\citep{reedPierce:icfp10}.  If a lambda abstraction $\lam x t$ is
typed with $\qfun r P N$, qualifier $r$ is attached to variable $x$ in
the resource context (rule \rintro\lolli).  If a function
$t : \qfun r P N$ is applied (rule \relim\lolli), we need an
$r$-qualified argument $v : rP$.  We could have given a qualified
value typing judgement $\qValTy \gamma \Gamma v {rP}$ in the style of
\citet{mcBride:wadler60} that---in the quantitative
interpretation---provides $r$ copies of $v$ to be consumed by the
function.  Such a judgement could come with a scaling rule
\[
\ru{\qValTy \gamma \Gamma v {sP}
  }{\qValTy{r\gamma}\Gamma v {(rs)P}}
\]
that allows to scale the production of $v$ by $r$ if the resources are
scaled accordingly (from $\gamma$ to $r\gamma$).
%
However, this would have prevented the typing of $\force v$ (rule
\relim\Box), bcause there is no place for the scaling factor in
computation typing.  Semantically, the $v$ in $\force v$
lives in $\D[\_]\denn N$, and
we can only $\textract$ the computation in $\denn N$ if we can
instantiate the coeffect qualifier $\_$ to $1$.  Such is not possible
if scaling already happened (and needs to be respected).
%
Instead, we bake scaling
into the transition from values to computations.
Thus, $\relim\lolli$ receives an
argument $\qValTy \delta \Gamma v P$, and to satisfy the demands of
the function $\qCompTy \gamma \Gamma t {\qfun r P N}$, the resources
$\delta$ for the argument are scaled by $r$, summing the resource
requirements for the application to $\gamma + r\delta$.
%
Analogous scaling of the eliminatee is baked into the other value
eliminators (%\relim\Box,
\relim\GS, \relim\otimes), into %the \rlet-rules,
 \rlet,
and into \rintro\diamond.

The monadic type $\compty r P$ records the multiplicity $r$ of the
value of type $P$ resulting from the computation.  This construction
is dual to the comonadic type $\thunkty e N$ from effect-graded CBPV
(Section~\ref{sec:effect-cbpv}).  Since negative types are not
$\D$-coalgebras and do not support scaling, the scaling in
$\rintro\diamond$ is the last opportunity for scaling before entering
the monad.  Typing of bind ($\bind x u t$) attaches the resource qualifier $r$ store in $\compty r P$ to the variable $x$ (\relim\diamond).


% As a consequence of eliminatee scaling, $\relim\Box$ is no longer
% ``natural-deduction'' style like projection, but ``sequent-calculus''
% style like $\relim\diamond$.  The original construct $\force v$ could
% eliminate an unscaled thunk $\qValTy \gamma \Gamma v {\Box N}$,
% however, computation typing $\qCompTy \gamma \Gamma {\force v} N$ has
% no scaling option, since negative types $N$ will not be interpreted as
% $\D$-coalgebras.  Scaling is semantically restricted to positive
% types, and in our type system, to context entries.  Thus, we extend
% CBPV to negative variables $y : N$ that come with a resource qualifier
% $r$ in the context and can there be interpreted as elements of
% $\D[r]\,\denn N$.  This is a modification also taken in call-by-need
% CBPV \citep{mcDermottMycroft:esop19}; this compromise allows us to
% keep CBPV's polarization of types where positive types stand for
% comonad coalgebras and negative types for monad algebras.


\subsection{Coeffect-graded CBPV: denotational semantics}


%\subsubsection{Interpretation of types}

Positive types $P$ are interpreted as $\D$-coalgebras
$\denp P : [(\R,\leq) \to \CC]$ and negative types $N$ as objects
$\denn N : \CC$.
% , except for in contexts, where they are interpreted as
% $\D$-algebras $\Den N r = \D[r]\,\denn N$.  With $\den P = \denp P$,
% the interpretation of contexts is thus
% $\den \Gamma = \boxtimes_{x:\dom\Gamma} \den{\Gamma(x)}$.
The interpretation of contexts is
$\den \Gamma = \boxtimes_{x:\dom\Gamma} \denp{\Gamma(x)}$.
\[
\begin{array}{lcl}
  \Denp\_\_ & : & \PTy \to [(\R,\leq) \to \CC] \\
  \Denp{\Box N}    r & = & \D[r]\, \denn N \\
  \Denp{\sumt I P} r & = & \coprod_{i:I} \Denp{P_i} r \\
  \Denp{\tupt I P} r & = & \otimes_{i:I} \Denp{P_i} r \\
\end{array}
\qquad\qquad
\begin{array}{lcl}
  \denn\_\_ & : & \NTy \to \CC \\
  \denn{\compty r P} & = & \T \Denp P r \\
  \denn{\qfun r P N} & = & \Denp P r \lolli \denn N \\
  \denn{\rect I N}   & = & \prod_{i:I} \denn{N_i} \\
\end{array}
\]
The symmetric monoidal category $\CC$ needs to be equipped with cartesian
products $\rect I B$ as well as with distributive coproducts
$\coprod_I A$ with distribution morphism
$\tdist : \Hom {C \otimes \coprod_I A} {\coprod_{i:I} {(C \otimes A_i)}}$.

Computations $\qCompTy \gamma \Gamma t N$ are interpreted as morphisms
$\dent t : \Hom {\Den\Gamma\gamma} {\denn N}$ and values
$\qValTy \gamma \Gamma v P$ as families
$\denv v r : \Hom {\Den\Gamma{r\gamma}} {\Denp P r}$ natural in $r$.
Naturality here means that
$\Denp P {r \leq s} \comp \denv v r = \denv v s \comp \left(\Den \Gamma {(\_
    \cdot \gamma)}\right)_{r \leq s}$.
Again, because of $\rweak$, we interpret typing derivations rather than terms.
The interpretation is now rather straightforward, but we spell it out for reference.
\[
\begin{array}{llcl}
\rvar %^+
  & \denv {\ValTy{\qext 0 \Gamma 1 x P} x P} r
  & = & \Gr \comp (\tdrop[\den\Gamma] \otimes \tid[\Denp P r])
\\
\rintro\Box
  & \denv {\qValTy \gamma \Gamma {\thunk t} {\Box N}} r
  & = & \fmapt[{\D[r]}]{\dent t} \comp \expose[\den \Gamma]
\\
\rintro\GS
  & \denv{\qValTy \gamma \Gamma {\inj i v} {\sumt I P}} r
  & = & \iota_i \comp \denv v r
\\
\rintro\otimes
  & \denv{\pValTy {\sumt I \gamma} \Gamma {\tup v} {\tupt I P}} r
  & = & \tupt {i:I} \denv {v_i}r \comp \duplicate
\\
\end{array}
\]
\[
\begin{array}{llcl}
% \rvar^-
%   & \dent {\oCompTy{\qext 0 \Gamma 1 y N} y N}
%   & = & \Gr \comp (\tdrop[\den\Gamma] \otimes \extract[\denn N])
% \\
% \rlet^{-}
%   & \dent {\pCompTy {\gamma + r\delta} \Gamma {\qlet r y u t} {N'}}
%   & = & \dent t
%   \comp (\tid[\Den \Gamma \gamma] \otimes
%            (\fmapt[{\D[r]}] \dent u \comp \expose[\den \Gamma]))
% \\ &&& {~}
%   \comp \duplicate
% \\
\rlet %^{+}
  & \dent {\pCompTy {\gamma + r\delta} \Gamma {\qlet r x v t} {N}}
  & = & \dent t
  \comp (\tid[\Den \Gamma \gamma] \otimes \denv v r)
  \comp \duplicate
\\
\relim\Box
  & \dent {\qCompTy \gamma \Gamma {\force v} {N}}
  & = & \extract[\D] \comp \denv v 1
% \relim\Box
%   & \dent {\pCompTy {\gamma + r\delta} \Gamma {\qforced r v y t} {N'}}
%   & = & \dent t
%   \comp (\tid[\Den \Gamma \gamma] \otimes \denv v r)
%   \comp \duplicate
\\
\relim\GS
  & \dent {\pCompTy {\gamma + r\delta} \Gamma {\caser v \{r x_i.\, t_i\}_{i:I}} {N}}
  & = & [\dent {t_i}]_{i:I} \comp \tdist
  \comp (\tid[\Den \Gamma \gamma] \otimes \denv v r)
  \comp \duplicate
\\
\relim\otimes
  & \dent {\pCompTy {\gamma + r\delta} \Gamma {\qsplits r v {\bar x} t} {N}}
  & = & \dent t \comp \isot
  \comp (\tid[\Den \Gamma \gamma] \otimes \denv v r)
  \comp \duplicate
\\
\rintro\diamond
  & \dent {\pCompTy {r\gamma} \Gamma {\ret v} {\compty r P}}
  & = & \treturn \comp \denv v r
\\
\relim\diamond
  & \dent {\pCompTy {\gamma + r\delta} \Gamma {\qbind r x u t} {N}}
  & = & \run[\den N] \comp \fmapt[\T]{\dent t} \comp \strengthl
\\ &&& {~}
  \comp (\tid[\Den \Gamma \gamma] \otimes \dent u)
  \comp \duplicate
\\
\rintro\lolli
  & \dent {\qCompTy \gamma \Gamma {\lam x t} {\qfun r P N}}
  & = & \Lambda \dent t
\\
\relim\lolli
  & \dent{\pCompTy {\gamma + r\delta} \Gamma {\app t v} N}
  & = & \teval \comp (\dent t \otimes \denv v r) \comp \duplicate
\\
\rintro\Pi
  & \dent {\qCompTy \gamma \Gamma {\recrd I t} {\rect I N}}
  & = & \langle \dent {t_i} \rangle_{i:I}
\\
\relim\Pi
  & \dent {\qCompTy \gamma \Gamma {\proj i t} {N_i}}
  & = & \pi_i \comp \dent t
\\
\rweak
  & \dent {\qCompTy \gamma \Gamma t N}
  & = & \dent t \comp \Den \Gamma {\gamma \leq \gamma'}
\\
\end{array}
\]

It is remarkable that coeffect-graded CBPV works \emph{without any
  distributive law} \cite{orchard:icfp16}
for the monad $\T$ and the graded comonad $\D$.
%
This was a crucial design criterion, leading to resource qualifiers in
the monadic type $\compty r P$.
%
A distributive law would be required if we allowed scaling of
computations, not just of values.


\subsection{Coeffect-graded CBPV: equational theory and operational semantics}

The equational theory and operational semantics of coeffect-graded
CBPV is identical to the one of CBPV \citep[Fig.~11]{levy:hosc06}.  In
this section, we make just a few remarks on an alternative
presentation of the permutation laws (called ``sequencing laws'' in
\loccit).

The equational theory of pure CBPV comprises $\beta$ and $\eta$ laws,
and \emph{sequencing laws} that syntactically express that negative
types are monad algebras \citep[Fig.~11]{levy:hosc06}.  The sequencing
laws contain a generalization of the associativity law for monads and
allow to permute a bind under a $\lambda$ or record construction.  In
the presence of $\eta$ for functions and records, they are
inter-derivable with the following permutation laws:
\[
\begin{array}{llcl}
\rpi\diamond
  & \bind {x_2} {(\bind {x_1} {t_1} {t_2})} {t_3}
  & = & \bind {x_1} {t_1} {(\bind {x_2} {t_2} {t_3})}
\\
\rpi\lolli
  & \app {(\bind x u t)} v
  & = & \bind x u {\app t v}
\\
\rpi\Pi
  & \proj i {(\bind x  u t)}
  & = & \bind x u {\proj i t}
\end{array}
\]
Note that permutations for let-bindings like
$\app{(\qlet r x v t)}w = \qlet r x v {\app t w}$ are instances of the
$\beta$-law $\qlet r x v t = \qsubst v r x t$.
%
Herein, \fbox{$\qsubst v r x t$} shall the denote the
(capture-avoiding) replacement $\subst v x t$ of a positive variable
$x$ annotated by $r$ by value $v$ in term $t$.
%
Permutations for value-eliminations like
$\app {(\qsplits r v {\bar x} t)} w = \qsplits r v {\bar x} {\app t
  w}$
are derivable (using the $\beta$-laws) from the $\eta$-laws, like
$\qsubst{v}{r}{z}{t'} = \qsplits r {v} {\bar x} {\qsubst {\ptup {\bar
      x}} r z {t'}}$ for $\otimes$.

% Our extension of CBPV features negative variables $y$ standing for
% computations $u$, thus, we require a notion of computation
% substitution $\qsubst u r t y$ that should have the following semantics
% (see corollaries \ref{cor:tysinglesub} and \ref{cor:semsinglesub}).
% \begin{align*}
% &
%   \ru{\qCompTy \delta \Gamma u N \qquad
%       \oCompTy {\qext \gamma \Gamma r y N} t {N'}
%     }{\pCompTy {\gamma + r\delta} \Gamma {\qsubst u r y t} {N'}}
% \\ &
%   \dent{\qsubst u r y t} = \dent t
%     \comp (\tid[\den\Gamma \gamma]
%              \otimes (\fmapt[{\D[r]}] \dent u \comp \expose[\den\Gamma]))
%     \comp \duplicate[\den\Gamma]
% \end{align*}

% Further, we add $\beta$-laws for the negative
% let-binding and $\relim\Box$, since $\tforce$ now has let-style elimination:
% \[
% \begin{array}{llcl}
% \rbeta{\rlet^{-}}
%   & \qlet r y u t
%   & = & \qsubst u r y t
% \\
% \rbeta\Box
%   & \qforced r {(\thunk u)} y t
%   & = & \qsubst u r y t
% \\
% \end{array}
% \]
% These laws are immediately sound, since left and right hand sides of
% these equations have the same denotation.
% %
% Read as reduction rules, these laws are added to the operational
% semantics of CBPV, replacing $\beta$-law $\force{(\thunk u)} = u$.


\subsection{Coeffect-graded CBPV: substitution and metatheory}
\label{sec:subst}

Substitution for coeffect-graded type system has been worked
out in detail by \citet{atkeyWood:types19,woodAtkey:linearity20}.
It straightforwardly extends to coeffect-graded CBPV.
%
A substitution $\sigma$ is a finite map from variable to terms,
mapping positive variables $x$ to values $v$ and negative variables
$y$ to computations $t$.  Substitution typing
\fbox{$\SubstTy \Psi \Gamma \sigma \Delta$} is equipped with a matrix
$\Psi : \R^{\dom\Gamma \times \dom\Delta}$ recording the usage
vector $\Psi(z) : \R^{\dom\Gamma}$ for each variable
$z \in \dom\Delta = \dom \sigma$ in the domain of the substitution.
\[
  \ru{
    }{\SubstTy {0} \Gamma \sempty \cempty}
\qquad
  \ru{\SubstTy \Psi \Gamma \sigma \Delta \qquad
      \qValTy \gamma \Gamma v P
    }{\pSubstTy {\Psi,\gamma} \Gamma {\sext \sigma x v} {\ext \Delta x P}}
% \qquad
%   \ru{\SubstTy \Psi \Gamma \sigma \Delta \qquad
%       \qCompTy \gamma \Gamma t N
%     }{\pSubstTy {\Psi,\gamma} \Gamma {\sext \sigma y t} {\ext \Delta y N}}
\]
Capture-avoiding parallel substitution into values \fbox{$v \sigma$} and
computations \fbox{$t \sigma$} is defined as usual by recursion on the term.

The matrix $\Psi : \R^{\dom \Gamma \times \dom \Delta}$ acts as a
linear map
$(\delta : \R^{\dom \Delta}) \mapsto (\delta\Psi : \R^{\dom \Gamma})$
and allows us to state the substitution theorem.
\begin{theorem}[Substitution preserves typing]
  Let $\SubstTy \Psi \Gamma \sigma \Delta$.
  \begin{enumerate}
  \item If $\qValTy \delta \Delta v P$ then $\pValTy {\delta\Psi} \Gamma {v\sigma} P$.
  \item If $\qCompTy \delta \Delta t N$ then
    $\pCompTy {\delta\Psi} \Gamma {t\sigma} N$.
  \end{enumerate}
\end{theorem}
\begin{proof}
  Straightforward adaptation of \citet{atkeyWood:types19,woodAtkey:linearity20}.
\end{proof}
\begin{corollary}[Single substitution preserves typing]
  \label{cor:tysinglesub}
  \bla
  \begin{enumerate}
  \item If $\qValTy \delta \Gamma v P$ and
      $\ValTy {\qext \gamma \Gamma r x P} w {P'}$
      then $\pValTy {\gamma + r\delta} \Gamma {\qsubst v r x w} {P'}$.
  \item If $\qValTy \delta \Gamma v P$ and
      $\oCompTy {\qext \gamma \Gamma r x P} t {N}$
      then $\pCompTy {\gamma + r\delta} \Gamma {\qsubst v r x t} {N}$.
  % \item If $\qCompTy \delta \Gamma u N$ and
  %     $\oCompTy {\qext \gamma \Gamma r y N} t {N'}$
  %     then $\pCompTy {\gamma + r\delta} \Gamma {\qsubst u r y t} {N'}$.
  \end{enumerate}
\end{corollary}
% %\begin{remark}
% Analogous statements are available for the substitution into values
% $\ValTy {\qext \gamma \Gamma r x P} w {P'}$ and
% $\ValTy {\qext \gamma \Gamma r y N} w {P}$, however, they are not
% invoked in the proof of subject reduction.
% %\end{remark}
\begin{corollary}[Subject reduction]
  $\beta$-reduction of coeffect-graded CBPV preserves types.
\end{corollary}
In the pure (effect-free) version, every computation at value type
returns a value.
\begin{conjecture}[Canonicity of pure coeffect-graded CBPV]
  If $\vdash t : \diamond P$ then $t =_{\beta} \ret v$ for some $\vdash v : P$.
\end{conjecture}
This normalization result can be proved with standard techniques such
as reducibility candidates or normalization by evaluation (\cf
\citet{abelSattler:ppdp19}).

The denotation of a substitution $\SubstTy \Psi \Gamma \sigma \Delta$
is a natural transformation
$\denv \sigma \delta : \Hom {\Den \Gamma {\delta\Psi}} {\Den \Delta
  \delta}$
such that
$
\Den \Delta {\delta \leq \delta'}
\comp
\denv \sigma \delta
=
\denv \sigma {\delta'}
\comp
\Den \Gamma {\delta\Psi \leq \delta'\Psi}
$.
It is defined by recursion as follows:
\[
\begin{array}{ll@{~}c@{~}l@{~}c@{~}l}
\ru{
    }{\SubstTy {0} \Gamma \sempty \cempty}
& \denv \sempty \delta
& : & \Hom {\Den\Gamma 0} \I
\\ &
& = & \tdrop[\den\Gamma]
\\[3ex]
\ru{\SubstTy \Psi \Gamma \sigma \Delta \qquad
      \qValTy \gamma \Gamma v P
    }{\pSubstTy {\Psi,\gamma} \Gamma {\sext \sigma x v} {\ext \Delta x P}}
& \denvp {\sext \sigma x v} {\delta,r}
& : & \Hom  {\Den\Gamma{(\delta,r)(\Psi,\gamma)}} {\Den \Delta \delta \otimes \Denp P r}
\\ &
& = & (\denv \sigma \delta \otimes \denv v r)
\comp \duplicate[\den\Gamma]
% \\[3ex]
% \ru{\SubstTy \Psi \Gamma \sigma \Delta \qquad
%       \qCompTy \gamma \Gamma t N
%     }{\pSubstTy {\Psi,\gamma} \Gamma {\sext \sigma y t} {\ext \Delta y N}}
% & \denvp {\sext \sigma y t} {\delta,r}
% & : & \Hom  {\Den\Gamma{(\delta,r)(\Psi,\gamma)}} {\Den \Delta \delta \otimes \D[r]\,\denn N}
% \\ &
% & = & (\denv \sigma \delta \otimes (\fmapt[{\D[r]}]{\dent t} \comp \expose[{\den\Gamma}]))
% \comp \duplicate[\den\Gamma]
\end{array}
\]
Observe that $(\delta,r)(\Psi,\gamma) = \delta\Psi + r\gamma$ and,
consequently,
$\duplicate[\den\Gamma] : \Hom {\Den\Gamma{(\delta,r)(\Psi,\gamma)}}
{\Den\Gamma{\delta\Psi} \otimes \Den\Gamma{r\gamma}}$.
\begin{theorem}[Soundness of substitution]
  Let $\SubstTy \Psi \Gamma \sigma \Delta$.
  \begin{enumerate}
  \item If $\qValTy \delta \Delta v P$ then
     $\denv{v\sigma}r = \denv v {r} \comp \denv \sigma {r\delta}
      : \Hom {\Den{\Gamma}{r\delta\Psi}} {\Denp P r}$.
  \item If $\qCompTy \delta \Delta t N$ then
     $\dent{t\sigma} = \dent t \comp \denv \sigma \delta
      : \Hom {\Den\Gamma{\delta\Psi}} {\denn N}$.
  \end{enumerate}
\end{theorem}
\begin{proof}
  By induction on the typing derivation of the term $v$ and $t$, resp.
\end{proof}
\begin{corollary}[Soundness of single substitution]
  \label{cor:semsinglesub}
  \bla
  \begin{enumerate}
  \item If $\qValTy \delta \Gamma v P$ and
      $\oCompTy {\qext \gamma \Gamma r x P} w {P'}$
      then
\[
  \denv {\qsubst v r x w} s = \denv w s
    \comp (\tid[\Den\Gamma{s\gamma}] \otimes \denv v {sr})
    \comp \duplicate[\den\Gamma]
  .
\]
  \item If $\qValTy \delta \Gamma v P$ and
      $\oCompTy {\qext \gamma \Gamma r x P} t {N}$
      then
\[
  \dent{\qsubst v r x t} = \dent t
    \comp (\tid[\Den\Gamma \gamma] \otimes \denv v r)
    \comp \duplicate[\den\Gamma]
  .
\]
%   \item If $\qCompTy \delta \Gamma u N$ and
%       $\oCompTy {\qext \gamma \Gamma r y N} t {N'}$ then
% \[
%   \dent{\qsubst u r y t} = \dent t
%     \comp (\tid[\Den\Gamma \gamma]
%              \otimes (\fmapt[{\D[r]}] \dent u \comp \expose[\den\Gamma]))
%     \comp \duplicate[\den\Gamma]
%   .
% \]
  \end{enumerate}
\end{corollary}
\begin{proof}
  After comprehending that the semantics of identity substitutions is
  just $\tid$, we observe that the semantics of single substitutions
  is $\dent {v/rx} = (\tid \otimes \denv v r) \comp \duplicate$.
% and
%   $\dent {t/ry} = (\tid \otimes (\fmapt[{\D[r]}]{\dent t} \comp \expose[{\den\Gamma}])) \comp
%   \duplicate$, resp.
\end{proof}


% \subsection{Coeffect-graded CBPV without negative variables}

% We introduce negative variables $y : N$ for the sake of grading thunking ($\Box$).  There is an alternative


\section{Fully graded CBPV}
\label{sec:full-cbpv}

\newcommand{\peCompTy}[5]{(#1)#2 \vdash #3 : #5 \mid #4}
\newcommand{\qeCompTy}[5]{#1#2 \vdash #3 : #5 \mid #4}
% \newcommand{\eext}[4]{#1.#2{:}#4#3}
% \newcommand{\qeext}[6]{#1#2.#3#4{:}#6#5}
\newcommand{\eext}[4]{#1\mathbin.#2:#4|#3}
\newcommand{\qeext}[6]{#1#2\mathbin.#3#4:#6|#5}

\newcommand{\gCompTy}[4]{\CompTy{#1}{#2}{\graybox{#3}}{#4}}
\newcommand{\gpeCompTy}[5]{(#1)#2 \vdash #3 : #5 \mid {\graybox{#4}}}
\newcommand{\gqeCompTy}[5]{#1#2 \vdash #3 : #5 \mid {\graybox{#4}}}
\newcommand{\gthunkty}[2]{\thunkty{\graybox{#1}}{#2}}
\newcommand{\geext}[4]{\eext{#1}{#2}{\graybox{#3}}{#4}}
\newcommand{\gqeext}[6]{\qeext{#1}{#2}{#3}{#4}{\graybox{#5}}{#6}}

In this section we finally present a \emph{fully graded} version of
CBPV where both effects and coeffects are graded.
Fig.~\ref{fig:full-cbpv} presents its types and typing rules where the
differences to the coeffect-graded version are
\graybox{\mbox{highlighted}}.

% We reintroduce effect grading into the computation typing
% $\qeCompTy \gamma \Gamma t e N$.  As in the effect-graded version,
% effects are accumulated in a single place, the monadic bind construct
% (\relim\diamond).  The negative variables $y : N$ we have introduced for
% coeffect grading are now annotated with an effect qualifier $e$ in the
% typing context $\Gamma$.  The full information about a negative
% variable $y$ in a graded context $\gamma\Gamma$ is type $N$, effect
% $e$ and coeffect $r$.  This combination is interpreted as
% $\D[r]\,\Denn N e$ where the resource qualifier acts as ``multiplier''
% for the effect qualifier $e$.  However, we never have to carry out
% this ``multiplication'', thus, do not require a law how the comonad
% $\D$ distributes over the monad algebra $N$.  The reason is that only
% variables $y$ with qualifier $r=1$ can be used from the context $(\rvar^{-})$.
% In a well-formed derivation, other resource qualifiers for variables are broken down
% in the course of the derivation until at the leaves (\rvar) only a
% single resource qualifier $r=1$ is left.

% The crucial insight concerning the absence of distributive laws is the
% laziness of effects in CBPV.  They are only observed at positive
% types \citep{levy:hosc06},
% thus, using a negative variable $y : N \mid e$ only schedules
% the effect $e$.  Likewise, in a negative let-binding ($\rlet^{-}$),
% the effects $e$ associated with the computation $u$ bound to $y$ are
% not run but ``bottled up'' with the variable in the context.  The same
% happens for thunk elimination ($\relim\Box$): the effects of the
% forced value $v : \thunkty e N$ are stored with the variable $y$
% holding the contents of the thunk $v$.

\begin{figure}[htbp]
\flushleft
\ruler{}
Types.
\[
\begin{array}{lllrl@{\qquad}l}
\PTy & \ni & P
  & ::= & \gthunkty e N
     \mid o \mid \sumty i I P \mid \tupty i I P
  & \nbox{Value types (positive types)} \\
\NTy & \ni & N
  & ::= & \compty r P
     \mid \qfun r P N \mid \recty i I N
  & \nbox{Computation types (negative types)} \\
% \Cxt & \ni & \Gamma
%   & ::= & \cempty \mid \ext \Gamma x P % \mid \geext \Gamma y e N
%   & \nbox{Typing context}
% \\
% \R & \ni & \graybox{r,s}
%    & ::= & 0 \mid 1 \mid r + s \mid rs
%    & \mbox{Resource qualifiers}
% \\
% \RCxt & \ni & \graybox{\gamma,\delta}
%   & ::= & \cempty \mid \ext \gamma x r \mid \ext \gamma y r
%   & \mbox{Resource context}
\end{array}
\]
% \dashruler{}
% Terms.
% \[
% \begin{array}{lllrl@{\qquad}l}
% \PTm & \ni & v,w
%   & ::= & x
%      \mid \thunk t
%      \mid \inj i v
%      \mid \ptup{\bar v}
%   & \nbox{Values (positive terms)} \\
% \NTm & \ni & t,u
%   & ::= & y \mid \qlet r y u t
%      \mid \qlet r x v t
%   & \nbox{Computations (negative terms):}
% \\ &&& \mid &
%          \qforced r v y t
%     \mid \qcase r v {\overline{x.\,t}}
%     \mid \qsplits r v {\bar x} t
%   & \nbox{value eliminations}
% \\ &&& \mid &
%          \ret v
%     \mid \bind x u t
%   & \nbox{monad operations}
% \\ &&& \mid &
%          \lam x t      \mid \app t v
%   & \nbox{functions}
% \\ &&& \mid &
%          \trecord\{\overline{i:t}\}
%        \mid \proj i t
%   & \nbox{lazy tuples (records)}
% \end{array}
% \]
\dashruler{}
Value typing \fbox{$\qValTy \gamma \Gamma v P$}\,.
\begin{gather*}
 \nru{\rvar %^+
    }{
    }{\ValTy {\qext 0\Gamma 1x P} x P}
\qquad
 \nru{\rintro\Box}
     {\gqeCompTy \gamma \Gamma t e N
    }{\qValTy \gamma \Gamma {\thunk t} {\gthunkty e N}}
\\[1.5ex]
 \nru{\rintro\GS}
     {\qValTy \gamma \Gamma v {P_i}
    }{\qValTy \gamma \Gamma {\inj i v} {\sumt I P}}
\qquad
 \nru{\rintro\otimes}
     {\forall i{:}I, \ \qValTy {\gamma_i} \Gamma {v_i} {P_i}
    }{\pValTy {\GS_{i:I}\gamma_i} \Gamma {\ptup v} {\tupt I P}}
\end{gather*}
\dashruler{}
Computation typing \fbox{$\gqeCompTy \gamma \Gamma t e N$}\,.
%\vspace{-2ex}
\begin{gather*}
%  \nru{\rvar^-}{}
%      {\gCompTy {\gqeext 0 \Gamma 1 y e N} y e N}
% \qquad
%  \nru{\rlet^-}
%      {\gqeCompTy \delta \Gamma u e N \qquad
%       \gCompTy {\gqeext \gamma \Gamma r y e N} t {e'} {N'}
%     }{\gpeCompTy {\gamma + r\delta} \Gamma {\qlet r y u t} {e'} {N'}}
% \\[1.5ex]
 \nru{\rlet} %^+}
     {\qValTy \delta \Gamma v P \qquad
      \CompTy {\qext \gamma \Gamma r x P} t e N
    }{\peCompTy {\gamma + r\delta} \Gamma {\qlet r x v t} e N}
\qquad
 \nru{\relim\Box}
     {\qValTy \gamma \Gamma v {\gthunkty e N}
    }{\gqeCompTy \gamma \Gamma {\force v} e {N}}
 % \nru{\relim\Box}
 %     {\qValTy \delta \Gamma v {\gthunkty e N} \qquad
 %      \gCompTy {\gqeext \gamma \Gamma r y e N} t {e'} {N'}
 %    }{\gpeCompTy {\gamma + r\delta} \Gamma {\qforced r v y t} {e'} {N'}}
\\[1.5ex]
 \nru{\relim\GS}
     {\qValTy \delta \Gamma v {\sumt I P} \quad
      \forall i{:}I,\ \CompTy {\qext \gamma \Gamma r {x_i} {P_i}} {t_i} e N
    }{\peCompTy {\gamma + r\delta} \Gamma {\caser v {\{r x_i.\, t_i\}_{i:I}}} e N}
\quad
 \nru{\relim\otimes}
     {\qValTy \delta \Gamma v {\tupt I P} \quad
      \qeCompTy \gamma {\extr \Gamma {\overline{r x_i{:}P_i}^{i:I}}} t e N
    }{\peCompTy {\gamma + r\delta} \Gamma {\qsplits r v {\bar x} t} e N}
\\[1.5ex]
 \nru{\rintro\diamond}
     {\qValTy \gamma \Gamma v P
    }{\gqeCompTy {r\gamma} \Gamma {\ret v} \Ge {\compty r P}}
\qquad
 \nru{\relim\diamond}
     {\gqeCompTy \delta \Gamma u {e_1} {\compty r P} \qquad
      \gCompTy {\qext \gamma \Gamma r x P} t {e_2} N
    }{\gpeCompTy {\gamma + \delta} \Gamma {\bind x u t} {e_1 \bu e_2} N}
\\[1.5ex]
 \nru{\rintro\lolli}
     {\CompTy {\qext \gamma \Gamma r x P} t e N
    }{\qeCompTy \gamma \Gamma {\lam x t} e {\qfun r P N}}
\qquad
 \nru{\relim\lolli}
     {\qeCompTy \gamma \Gamma t e {\qfun r P N} \qquad
      \qValTy \delta \Gamma v P
    }{\peCompTy {\gamma + r\delta} \Gamma {\app t v} e N}
\\[1.5ex]
 \nru{\rintro\Pi}
     {\forall i{:}I,\
      \qeCompTy \gamma \Gamma {t_i} e {N_i}
    }{\qeCompTy \gamma \Gamma {\trecord {\{i : t_i\}_{i:I}}} e {\rect I N}}
\qquad
 \nru{\relim\Pi}
     {\qeCompTy \gamma \Gamma t e {\rect I N}
    }{\qeCompTy \gamma \Gamma {\proj i t} e {N_i}}
\\[1.5ex]
 \nrux{\graybox{\rsub}}
      {\gqeCompTy {\gamma'} \Gamma t e N}
      {\gqeCompTy \gamma \Gamma t {e'} N}
      {\gamma \leq \gamma', \graybox{e \leq e'}}
\end{gather*}
\rule{\textwidth}{0.2pt}
  \caption{Fully graded call-by-push-value.}
  \label{fig:full-cbpv}
\end{figure}


We reintroduce effect grading into the computation typing
$\qeCompTy \gamma \Gamma t e N$.  As in the effect-graded version,
effects are accumulated in a single place, the monadic bind construct
(\relim\diamond).

The crucial insight concerning the absence of distributive laws is the
laziness of effects in CBPV.  They are only observed at positive
types \citep{levy:hosc06},
thus, can be ``bottled up'' in $\thunk t : \thunkty e N$ and
reactivated via forcing in $\force v : N \mid e$.


\subsection{Graded CBPV: semantics}

As in effect-graded CBPV, negative types $N$ are interpreted as graded
monad algebras $\denn N : [(\Eff,\leq) \to \CC]$, and as in
coeffect-graded CBPV, positive types are interpreted as graded comonad
coalgebras $\denp P : [(\R,\leq) \to \CC]$.  Values
$\qValTy \gamma \Gamma v P$ are interpreted as natural transformations
$\dent v : \Denpar \Gamma {\_\cdot\gamma} \todot \denp P$ and
computations $\qeCompTy \gamma \Gamma t e N$ as morphisms
$\dent t : \Hom {\Den\Gamma \gamma} {\Denn N e}$.  Except for some
change in typing, the denotation of terms is unchanged from
coeffect-graded CBPV.  We recapitulate the most interesting cases here
with the updated typing:
\[
\xymatrix@C=10ex@R=3ex{
\denv {\thunk t} r
& \Den \Gamma {r\gamma}     \ar[r]^{\expose}
& \D[r]\,\Den\Gamma\gamma   \ar[r]^{\fmapt[{\D[r]}]{\dent t}}
& \D[r]\,\Denn N e
\\
\dent {\force v}
& \Den \Gamma \gamma    \ar[r]^{\denv v 1}
& \D[1]\,\Denn N e      \ar[r]^{\extract}
& \Denn N e
\\
\dent {\ret v}
& \Den \Gamma {r\gamma}  \ar[r]^{\denv v r}
& \Denp P r              \ar[r]^{\treturn}
& \T[\Ge]\,\Denp P r
\\
\dent {\bind x u t}
& \Den \Gamma {\gamma + \delta}                 \ar[r]^{\duplicate}
& \Den \Gamma \gamma \otimes \Den \Gamma \delta \ar[r]^{\tid \otimes \dent u}
& \Den \Gamma \gamma \otimes \T[e_1] \Denp P r  \ar[d]^{\strengthl}
\\
& \Denn N {e_1 \bu e_2}
& \T[e_1]\,\Denn N {e_2}                        \ar[l]_{\run[\denn N]}
& \T[e_1](\Den \Gamma \gamma \otimes \Denp P r) \ar[l]_{\fmapt[{\T[e_1]}] {\dent t}}
% \dent {\bind x u t}
% & \Den \Gamma {\gamma + \delta}                 \ar[r]^{\duplicate}
% & \Den \Gamma \gamma \otimes \Den \Gamma \delta \ar[r]^{\tid \otimes \dent u}
% & \Den \Gamma \gamma \otimes \T[e_1] \Denp P r  \ar[r]^{\strengthl}
% & \T[e_1](\Den \Gamma \gamma \otimes \Denp P r) \ar[r]^{\fmap {\dent t}}
% & \T[e_1]\,N_{e_2}                               \ar[r]^{\trun}
% & N_{e_1 \bu e_2}
}
\]
Subsumption (\rsub) is interpreted by pre- and post-composition with
the functorial actions of $\den \Gamma$ and $\denn N$:
\[
\xymatrix@C=10ex{
% \rsub
% &
\Den \Gamma {\gamma}   \ar[r]^{\Den\Gamma{\gamma \leq \gamma'}}
& \Den \Gamma {\gamma'}  \ar[r]^{\dent t}
& \Denn N e              \ar[r]^{\Denn N {e \leq e'}}
& \Denn N {e'}
}
\]

% \begin{align*}
% & \denv {\thunk t} r
% & &
% \xymatrix{
%   \Den \Gamma {r\gamma}     \ar[r]^{\extract}
% & \D[r]\,\Den\Gamma\gamma   \ar[r]^{\fmapt[{\D[r]}]{\dent t}}
% & \D[r]\,\Denn N e
% }
% \\
% \xymatrix{
%   \Den \Gamma \gamma    \ar[r]^{\denv v 1}
% & \D[1]\,\Denn N e      \ar[r]^{\extract}
% & \Denn N e
% }
% \\
% & \dent {\ret v}
% & &
% \xymatrix@C=8ex{
%   \Den \Gamma {r\gamma}  \ar[r]^{\denv v r}
% & \Denp P r              \ar[r]^{\treturn}
% & \T[\Ge]\,\Denp P r
% }
% \end{align*}

The operational semantics of graded CBPV is identical to the one of
CBPV, and the substitution typing of coeffect-graded CBPV
(Section~\ref{sec:subst}) can be mechanically transferred to fully
graded CBPV.

\subsection{Discussion}

Graded CBPV provides syntax to work with graded monads $\T[e]$ and
comonads $\D[r]$.  \emph{A priori}, one could have expected that these
constructs are directly reflected in the syntax of types, as
$\compty e P$ and $\thunkty r N$.  Surprisingly, the adaptation of
graded effect and coeffect typing to CBPV places the qualifiers in the
opposite way, as $\compty r P$ and $\thunkty e N$.  Given our
semantics of types as (co)monad (co)algebras, this apparent oddity has
a natural explanation: the qualifiers do not instantiate the grade in
the monad or comonad, but in the respective (co)algebra.  The
alternative concrete syntax $\diamond P_r$ and $\Box N_e$ would maybe
transport the intended semantics, $\T[-]{\Denp P r}$ and
$\D[-]{\Denn N e}$, more directly.

How does Graded CBPV work as a programming language?  A program is
usually a set of definitions and then an entrypoint in form of an
expression or simply the name of the main procedure.  In Graded CBPV,
the definitions are given as a sequence of let-bindings whose final
body serves as the entrypoint.  The entrypoint should be an expression
of type $\compty 1 P$, producing an observable value and thus, the
effects leading up to this result.  The expressions bound by the lets
will often be functions (or records), but wrapped in thunks
$\thunkty e N$ to satisfy the formal requirement to be of positive
type.  This thunk is where the effects are declared that a function
produces.  Thus, in the big picture, the effects of a computation
\emph{are} stated in the types, not just in the typing judgement
$\qeCompTy \gamma \Gamma t e N$, even though the effect annotation is
in a \emph{a priori} unexpected location.

\section{Related Work}

\subsection{Effect tracking in CBPV}

\citet{mcDermottMycroft:esop19} observe that CBPV can represent
call-by-name and call-by-value evaluation strategies but not
call-by-need.  As a remedy, they extend CBPV to ECBPV by variables
$\underline x : \diamond P$ that hold monadic values, plus a
corresponding let-binding construct.  They further present an
effect-graded version of ECBPV, drawing effect qualifiers from a
preordered monoid.  In contrast to our version, effects annotations
$f$ are stored at monadic types $\compty f P$.  Instead of our
judgement $\CompTy \Gamma t e N$ that accumulates effect traces $e$
and a semantic interpretation of computation types $N$ as graded monad
algebras, they define a syntactic operation $\compty f N$ that pushes
down effect qualifier $f$ to the monadic types $\compty f' P$ in $N$
where the effects are combined to $\compty{f \bu f'} P$.  This way of
handling effect grading seems equivalent to our approach, albeit it is
syntactic rather than semantic.

In previous work, \citet{mcDermottMycroft:ocs18} use coeffect typing
to track variable usage in call-by-need simply-typed lambda calculus
with conditionals.  Compared to the usual resource semiring $\R$
tailored to call-by-name usage analysis, their coeffect algebra is
more expressive: it can produce traces of variable uses, including
bound variables.  They show how graded effect tracking can be
simulated by the coeffect typing, \eg, for the effect of
non-determinism.  Naturally, they focus on operational semantics.

% Filinski’s M3L (MultiMonadic MetaLanguage) [11],
% whose latest version draws heavily on CBPV [12]

To accomodate several sorts of effects in one language (CBPV hosts
only a single effect type), \citet{kammarPlotkin:popl12} present MAIL,
the Multi-Adjunctive Intermediate Language, which is a version of CBPV
\emph{parameterized} by effect qualifiers.  In contrast to our
effect-graded version of CBPV, effect qualifiers are a parameter to
the type system as a whole, basically a \emph{mode} attached to the
typing judgement.  Via effect subsumption, one can switch to a
``wider'' mode.  However, as effects qualifiers are drawn from a
preordered set rather than a monoid, more detailed static effect
traces cannot be captured.  In their own words:
\begin{quotation}
  Instead of one kind of computation, for each effectset
  $\varepsilon \in E$, we have $\varepsilon$-computations
  $\mathsf{Comp}(\varepsilon)$ that can cause effects in
  \ensuremath{\varepsilon}. We view MAIL as multiple copies of CBPV,
  one for each \ensuremath{\varepsilon}, sharing the same values. One
  can translate between these different CBPVs by means of coercion
  [...]
\end{quotation}

% POPL12:
% Algebraic Foundations for Effect-Dependent Optimisations
% Ohad Kammar   Gordon D. Plotkin
% w
% - MAIL: Multi-Adjunctive Intermediate Language
% - Both \Box and \diamonad annotated with effects
% - No effect accumulation

\subsection{Coeffect typing}

Our style of coeffect typing is heavily influenced by
\citet{mcBride:wadler60} and \citet{atkey:lics18} who separate the
quantification of variable \emph{usage} $\gamma$ from the variable
\emph{declaration} in an ordinary typing context $\Gamma$.  This
division of labor removed a major obstacle in the integration of
linear and dependent types: the customary sorting of linear and
unrestricted variables into separate typing contexts, and the
dominance of context concatenation operations in the typing of
multiplicative connectives in linear logic, \eg:
\[
  \ru{\Delta_1 \vdash v_1 : A_1 \qquad
      \Delta_2 \vdash v_2 : A_2
    }{\Delta_1.\Delta_2 \vdash (v_1, v_2) : A_1 \otimes A_2}
\]
In contrast, the coeffect type systems of
\citet{reedPierce:icfp10},
\citet{orchard:icfp14},
\citet{brunel:esop14},
\citet{ghicaSmith:esop14}, and
\citet{orchard:icfp19}
keep context concatenation which does no harm in simply-typed settings
but seems to duplicate a mechanism already present in form of the
resource qualifier $0$.

Our treatment of substitution in coeffect-graded CBPV is based on
\citet{woodAtkey:linearity20}, who formalize (in Agda) substitution
for intuitionistic linear logic with a subexponential representing a
graded comonad.  They observe that instead of a preordered semiring, a
left skew semiring can be used as resource algebra.  In a skew
semiring, the laws involving multiplication only hold as inequalities,
not equalities.

\subsection{Interaction of effects and coeffects}

\citet{orchard:icfp16} investigate systematically the possible
interactions between semiring-graded coeffects and monoid-graded
effects, in form of distributive laws between the graded comonad and
monad.  We do not rely on such distributive laws in Graded CBPV.  The
strict separation of value and computation types and the placement of
the monad and comonad at the transition points makes interaction
optional.  How to integrate distributive laws into Graded CBPV is left
for future research.



\section{Conclusions}

In this article, we have demonstrated that CBPV can accommodate graded
effects and coeffects in a smooth way, keeping the term grammar
virtually unchanged and only adding effect and coeffect annotations at
the transition points between value and computation types.  The
surprising simplicity of our solution speaks for the design quality of
Levy's CBPV calculus \cite{levy:hosc06}.  For instance, the analogy to
intuitionistic linear logic---that lies at the heart of coeffect
typing---is already implicitly present in the separation of value
$(\otimes)$ and computation type products ($\Pi$).

For the semantics of Graded CBPV, we have evolved the concepts of
monad algebra and comonad coalgebra to their graded versions.

In future work, we would like to investigate whether the distributive
laws of \citet{orchard:icfp16} carry over to Graded CBPV, \eg, whether
and how a graded comonad can be distributed over a graded monad
algebra or, in the symmetric case, a graded monad distributes over a
graded comonad coalgebra.

%% Acknowledgments
\begin{acks}                            %% acks environment is optional
                                        %% contents suppressed with 'anonymous'
  %% Commands \grantsponsor{<sponsorID>}{<name>}{<url>} and
  %% \grantnum[<url>]{<sponsorID>}{<number>} should be used to
  %% acknowledge financial support and will be used by metadata
  %% extraction tools.

  This material is based upon work supported by the
  Swedish Research Council (Vetenskapsrådet)
  under Grant
  No.~2019-04216 \emph{Modal Dependent Type Theory}.

  % This material is based upon work supported by the
  % \grantsponsor{GS100000001}{National Science
  %   Foundation}{http://dx.doi.org/10.13039/100000001} under Grant
  % No.~\grantnum{GS100000001}{nnnnnnn} and Grant
  % No.~\grantnum{GS100000001}{mmmmmmm}.  Any opinions, findings, and
  % conclusions or recommendations expressed in this material are those
  % of the author and do not necessarily reflect the views of the
  % National Science Foundation.
\end{acks}


%% Bibliography
\bibliography{medium}


% %% Appendix
% \appendix
% \section{Appendix}

% Text of appendix \ldots

\end{document}
